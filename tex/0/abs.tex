
% ----------------------------------------------------------------------
%                   Latex File for Eoin Carley's PhD (2013)
% ----------------------------------------------------------------------

%Latext thesis template from Harish Bhanderi's PhD/MPhil template, then Uni Cambridge
% http://www-h.eng.cam.ac.uk/help/tpl/textprocessing/ThesisStyle/

%: Style file for Latex
% Most style definitions are in the external file PhDthesisPSnPDF.
% In this template package, it can be found in ./Latex/Classes/
\documentclass[a4paper, oneside, 12pt]{Latex/Classes/PhDthesisPSnPDF}

%Change "oneside" to "twoside" for final submission-grade thesis after viva/corrections

\makeatletter
\newcommand{\rmnum}[1]{\romannumeral #1}
\newcommand{\Rmnum}[1]{\expandafter\@slowromancap\romannumeral #1@}
\makeatother

\usepackage{lineno}
\usepackage{amsbsy}
\usepackage{xspace}
\usepackage{wtmmPkg}
\usepackage{natbib}
\usepackage{multirow}
\usepackage{graphicx}
\usepackage{paralist}
\usepackage{titlesec}
\usepackage{lscape}
\usepackage{quotchap}
\usepackage{epstopdf}
\usepackage{fancyhdr}
\usepackage{amsmath}
\usepackage{float}
\usepackage{afterpage}
\usepackage{rotating}
\usepackage{tikz}

%Added by SM 13 Sep 2011 to get backref to read ''Cited on page''
   \usepackage{Latex/StyleFiles/backrefx}
       \renewcommand{\backrefpagesname}{Cited on page~}
       \renewcommand{\backrefpagesnames}{Cited on pages~}

\newcommand{\BibTeX}{\textsc{Bib}\TeX}
\newcommand{\etal}{{\it et al.}}

% Definitions for equations
\newcommand{\arcsec}{^{\prime\prime}}
%\def\ion#1#2{#1$\;${\small\rm\@Roman{#2}}\relax}
\DeclareRobustCommand{\ion}[2]{%
\relax\ifmmode
\ifx\testbx\f@series
{\mathbf{#1\,\mathsc{#2}}}\else
{\mathrm{#1\,\mathsc{#2}}}\fi
\else\textup{#1\,{\mdseries\textsc{#2}}}%
\fi}

\newcommand{\subion}{ {_{ion}} }
\newcommand{\sube}{ {_{e}} }
\newcommand{\subj}{ {_{j}} }
\newcommand{\subi}{ {_{i}} }
\newcommand{\ji}{ {_{j,i}} }
\newcommand{\rt}{ {$R(T)$} }
\newcommand{\rlam}{ {$R(\lambda)$} }
\newcommand{\rsun}{R$_{\odot}$}
\newcommand{\rmd}{ {\ \mathrm d} }
\renewcommand{\vec}[1]{ {\mathbf #1} }
\newcommand{\uvec}[1]{ \hat{\mathbf #1} }
\newcommand{\pder}[2]{ \f{\partial #1}{\partial #2} }
\newcommand{\grad}{ {\bf \nabla } }
\newcommand{\curl}{ {\bf \nabla} \times}
\newcommand{\vol}{ {\mathcal V} }
\newcommand{\bndry}{ {\mathcal S} }
\newcommand{\dv}{~{\mathrm d}^3 x}
\newcommand{\da}{~{\mathrm d}^2 x}
\newcommand{\dl}{~{\mathrm d} l}
\newcommand{\dt}{~{\mathrm d}t}
\newcommand{\intv}{\int_{\vol}^{}}
\newcommand{\inta}{\int_{\bndry}^{}}
\newcommand{\avec}{ \vec A}
\newcommand{\ap}{ \vec A_p}
\newcommand{\bb}{ \vec B}
\newcommand{\jj}{ \vec j}
\newcommand{\rr}{ \vec r}
\newcommand{\xx}{ \vec x}

% Definitions for the journal names
\newcommand{\adv}{    {\it Advances in Space Research}}
\newcommand{\annG}{   {\it Annales Geophysicae}}
\newcommand{\aap}{    {\it Astronomy \& Astrophysics}}
\newcommand{\aaps}{   {\it Astronomy \& Astrophysics Supplemental}}
\newcommand{\aapr}{   {\it Astronomy \& Astrophysics Review}}
\newcommand{\ag}{     {\it Ann. Geophys.}}
\newcommand{\aj}{     {\it Astronomical Journal}}
\newcommand{\apj}{    {\it Astrophysical Journal}}
\newcommand{\apjs}{    {\it Astrophysical Journal Supplemental Series}}
\newcommand{\apjl}{   {\it Astrophysical Journal Letters}}
\newcommand{\apss}{   {\it Astrophysics \& Space Science}}
\newcommand{\cjaa}{   {\it Chinese Journal Astronomy \& Astrophysics}}
\newcommand{\gafd}{   {\it Geophysical and Astrophysical Fluid Dynamics}}
\newcommand{\grl}{    {\it Geophysical Research Letters}}
\newcommand{\ijga}{   {\it International Journal of Geomagnetism and Aeronomy}}
\newcommand{\jastp}{  {\it Journal of Atmospheric and Solar-Terrestrial Physics}}
\newcommand{\jgr}{    {\it Journal of Geophysical Research}}
\newcommand{\mnras}{  {\it Monthly Notices of the Royal Astronomical Society}}
\newcommand{\nat}{    {\it Nature}}
\newcommand{\pasp}{   {\it Publications of the Astronomical Society of the Pacific}}
\newcommand{\pasj}{   {\it Publications of the Astronomical Society of Japan}}
\newcommand{\pra}{    {\it Physical Review A}}
\newcommand{\pre}{    {\it Physical Review E}}
\newcommand{\solphys}{{\it Solar Physics}}
\newcommand{\sovast}{ {\it Soviet Astronomy}}
\newcommand{\ssr}{    {\it Space Science Reviews}}
\newcommand{\araa}{  {\it Annual Review of Astronomy \& Astrophysics}}
\newcommand{\memsai}{ {\it Memorie della Societa Astronomia Italiana}}
\newcommand{\zap}{ {\it Zeitschrift fur Astrophysik}}
\newcommand{\bain}{ {\it Bulletin of the Astronomical Institutes of the Netherlands}}
\newcommand{\planss}{ {\it Planet.~Space~Sci.}}%


%: Macro file for Latex
% Macros help you summarise frequently repeated Latex commands.
% Here, they are placed in an external file /Latex/Macros/MacroFile1.tex
% An macro that you may use frequently is the figuremacro (see introduction.tex)
% This file contains macros that can be called up from connected TeX files
% It helps to summarise repeated code, e.g. figure insertion (see below).

% insert a centered figure with caption and description
% parameters 1:filename, 2:title, 3:description and label
\newcommand{\figuremacro}[3]{
	\begin{figure}[htbp]
		\centering
		\includegraphics[width=1\textwidth]{#1}
		\caption[#2]{\textbf{#2} - #3}
		\label{#1}
	\end{figure}
}

% insert a centered figure with caption and description AND WIDTH
% parameters 1:filename, 2:title, 3:description and label, 4: textwidth
% textwidth 1 means as text, 0.5 means half the width of the text
\newcommand{\figuremacroW}[4]{
	\begin{figure}[htbp]
		\centering
		\includegraphics[width=#4\textwidth]{#1}
		\caption[#2]{\textbf{#2} - #3}
		\label{#1}
	\end{figure}
}

% inserts a figure with wrapped around text; only suitable for NARROW figs
% o is for outside on a double paged document; others: l, r, i(inside)
% text and figure will each be half of the document width
% note: long captions often crash with adjacent content; take care
% in general: above 2 macro produce more reliable layout
\newcommand{\figuremacroN}[3]{
	\begin{wrapfigure}{o}{0.5\textwidth}
		\centering
		\includegraphics[width=0.48\textwidth]{#1}
		\caption[#2]{{\small\textbf{#2} - #3}}
		\label{#1}
	\end{wrapfigure}
}

% predefined commands by Harish
\newcommand{\PdfPsText}[2]{
  \ifpdf
     #1
  \else
     #2
  \fi
}

\newcommand{\IncludeGraphicsH}[3]{
  \PdfPsText{\includegraphics[height=#2]{#1}}{\includegraphics[bb = #3, height=#2]{#1}}
}

\newcommand{\IncludeGraphicsW}[3]{
  \PdfPsText{\includegraphics[width=#2]{#1}}{\includegraphics[bb = #3, width=#2]{#1}}
}

\newcommand{\InsertFig}[3]{
  \begin{figure}[!htbp]
    \begin{center}
      \leavevmode
      #1
      \caption{#2}
      \label{#3}
    \end{center}
  \end{figure}
}


%%% Local Variables: 
%%% mode: latex
%%% TeX-master: "~/Documents/LaTeX/CUEDThesisPSnPDF/thesis"
%%% End: 


%Change this if compiling at home/office
%\graphicspath{{/Users/josephroche/Work/log_of_learning/images/}}
\graphicspath{{images/}}


%: --------------------------------------------------------------
%:                  FRONT MATTER: dedications, abstract,..
% --------------------------------------------------------------
\usepackage{setspace}


\begin{document}




%!TEX root = ../thesis.tex
%Adding the above line, with the name of your base .tex file (in this case "thesis.tex") will allow you to compile the whole thesis even when working inside one of the chapter tex files

\begin{abstracts} 

Coronal mass ejections (CMEs) are large-scale eruptions of magnetized plasma from the low solar atmosphere into interplanetary space. With energies of up to $10^{25}$\,J, they are the most energetic eruptive phenomena in the solar system and are also the driver of plasma shocks from the corona into the heliosphere. Despite many years of study, the nature of the forces governing their eruption, and the kinematical behavior of the resulting shock, remain poorly understood. In this thesis I will present the first accurate calculation of the magnitude of the total force on a CME. I will also present previously unobserved plasma shock behavior that sheds new light into the kinematical nature of CME-driven shocks in the corona.

In the past, measurement of the forces governing the propagation of CMEs have been hindered by highly uncertain estimates of the total mass of the ejection. The primary source of uncertainty is the unknown position and geometry of the CME, leading to an erroneous treatment of the Thomson scattering equations which are used to estimate the mass. Geometrical uncertainty on the CMEs position and size has primarily been due to observations of the eruption from a single vantage point. However, with the launch of the {\it Solar Terrestrial Relations Observatory (STEREO)}, the two viewpoints can be exploited to derive the CMEs position and size, ultimately resulting in mass uncertainty that is both reliably quantified and much reduced. Using the {\it STEREO} spacecraft, a CME on the 12 December 2008 was found to have a mass of $3.4\pm1.0\times10^{12}$\,kg, meaning the mass uncertainty was less than 30\%. This is a substantial improvement on previous uncertainties which were well above 50\%, or entirely unquantifiable. The much better mass estimates can then be combined with kinematical results that are also more reliable and hence lead to the first reliable quantification of the total force acting on the CME. The dynamics are described by an early phase of strong acceleration, dominated by a force of peak magnitude of $3.4\pm2.2\times10^{14}$\,N at $\sim$3\,$R_{\odot}$. Using the magnetohydrodynamics (MHD) equation of motion, the relative sizes of the forces at each stage in the CME propagation are estimated, revealing the Lorentz force is the largest source of CME acceleration early in its propagation. Quantification of the Lorentz force magnitude from observations has never been achieved in the past.


The second part of this thesis will involve an investigation into the behaviour of radio-bright plasma shocks occurring in the corona. CMEs often erupt at speeds in excess of the local magnetosonic wave speeds in the corona. Traveling in excess of Alfv\'{e}n Mach 1, they often drive shocks which can have a variety of observational manifestations, such as type II and III radio bursts, coronal bright fronts (CBFs), white-light enhancements, and the eventual in-situ detection of solar energetic particles. Despite such a variety of shock phenomena being observed for decades, the unifying physical mechanism between these phenomena remains unknown. This thesis will provide an analysis that uses extreme ultraviolet, radio, and white-light imaging of a solar eruptive event on 22 September 2011 to determine the properties of a CME-driven shock in the corona. The results show that a plasma shock with an Alfv\'{e}n Mach number of $2.4^{+0.7}_{-0.8}$ was coincident with a coronal bright front and an intense decametric radio burst generated by electrons with kinetic energies of 2-46\,keV (0.1-0.4\,c). This work provides new observational evidence to show that the relationship between CMEs, CBFs, and type II and III radio bursts is a coronal plasma shock. 



\end{abstracts}

% ---------------------------------------------------------------------- 



%: ----------------------- contents ------------------------

\setcounter{secnumdepth}{3} % organisational level that receives a numbers
\setcounter{tocdepth}{3}    % print table of contents for level 3
\tableofcontents            % print the table of contents
% levels are: 0 - chapter, 1 - section, 2 - subsection, 3 - subsection


%: ----------------------- list of figures/tables ------------------------

\listoffigures	% print list of figures
%\listoftables  % print list of tables


%: --------------------------------------------------------------
%:                  MAIN DOCUMENT SECTION
% --------------------------------------------------------------


\mainmatter
\topmargin = 3mm
\marginparwidth = 35pt
\textheight = 630pt
\textwidth = 500pt
\footskip = 10mm




% --------------------------------------------------------------
%:                  BACK MATTER: appendices, refs,..
% --------------------------------------------------------------
\appendix



%: ----------------------- bibliography ------------------------


\bibliographystyle{Latex/Classes/jmb}
\renewcommand{\bibname}{References} 

\footnotesize



%\end{footnotesize}


\end{document}
