%!TEX root = ../thesis.tex
%Adding the above line, with the name of your base .tex file (in this case "thesis.tex") will allow you to compile the whole thesis even when working inside one of the chapter tex files

\begin{abstracts} 

Coronal mass ejections (CMEs) are large-scale eruptions of magnetized plasma from the low solar atmosphere into interplanetary space. With energies of up to $10^{26}$\,J, they are the most energetic eruptive phenomena in the solar system and are also the driver of plasma shocks from the corona into the heliosphere. Despite many years of study, the nature of the forces governing their eruption, and the kinematical behavior of the resulting shock, remain poorly understood. This thesis will presents the first accurate calculation of the magnitude of the total force on a CME. I will also show a previously unseen plasma shock behavior that sheds new light into the kinematical nature of CME-driven shocks in the corona.

In the past, measurement of the forces governing the propagation of CMEs have been hindered by highly uncertain estimates of the total mass of the ejection. The primary source of uncertainty is the unknown position and geometry of the CME, leading to an erroneous treatment of the Thomson scattering equations which are used to estimate the mass. Geometrical uncertainty on the CMEs position and size has primarily been due to observations of the eruption from a single vantage point. However, with the launch of the STEREO spacecraft, the two viewpoints can be exploited to derive the CMEs position and size, ultimately resulting in mass uncertainty that is both reliably quantified and much reduced. These much better estimates for the mass can then be combined with kinematical results that are also more reliable and hence lead to the first reliable quantification of the total force acting on the CME. 

This thesis will present the method by which mass values derived from the STEREO coronagraphs, and the uncertainties reliably quantified. Combining this with a previous kinematical analysis, the mechanical energies and total force on the CME is derived. Using the magnetohydrodynamical equation of motion, the relative sizes of the forces at each stage in the CME propagation are estimated, revealing the Lorentz force is the largest source of CME acceleration early in its propagation. This analysis also leads to a reliable observational estimate of size of this Lorentz force.

CMEs often erupt at speeds in excess of the local MHD wave speeds in the corona. Traveling in excess of Mach 1, they often drive shocks which can have a variety of manifestations, from radio bursts to the propagation of bright pulse seen in extreme ultraviolet (EUV) images. Despite these myriad shock phenomena being observed for decades, the relationship between them remains unknown. Chapters X and Y of this thesis, will describe the construction of instrumentation to observe high time sampling spectroscopy of these radio bursts. These observations are combined with high cadence radio and EUV images to reveal the presence of a shock driven by the expansion of the CME flank that resulted in both the EUV pulse and radio burst. Furthermore, the radio spectra evidence for particle acceleration at this shock is presented, revealing the shock was capable of producing a bursty acceleration of near-relativistic electrons. This previously unseen behavior sheds new light on the physics governing radio burst generation and the relationship to CMEs and EUV pulses.

\end{abstracts}

% ---------------------------------------------------------------------- 
