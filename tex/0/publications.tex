%!TEX root = ../thesis.tex
%Adding the above line, with the name of your base .tex file (in this case "thesis.tex") will allow you to compile the whole thesis even when working inside one of the chapter tex files
\chapter{List of Publications}
\label{chapter:publications}


\begin{enumerate}

\item \textbf{Carley, E.~P.}, MacAteer, R.~T. J., \& Gallagher, P.~T.\\
``Coronal Mass Ejection Masses, Energies, and Force Estimates Using \emph{STEREO}'', \\
\emph{The Astrophysical Journal}, Volume 752, Issue 1, article id. 36, 8 pp. (2012).

\item Zucca, P., \textbf{Carley, E.~P.},  McCauley, J., Gallagher, P. T. ,Monstein, C., \& MacAteer, R.~T. J.,\\
``Observations of Low Frequency Solar Radio Bursts from the Rosse Solar-Terrestrial Observatory'', \\
\emph{Solar Physics}, Volume 280, Issue 2, pp.591-602. (2012).

\item \textbf{Carley, E.~P.}, Long, D.~M., \& Gallagher, P.~T.\\
``Shock Acceleration of Energetic Particles in the Solar Atmosphere'', \\
\emph{Some Journal}, Volume X, Issue Y, article id. (2013)

\item Zucca, P., \textbf{Carley, E.~P.}, Bloomfield, S.~D., \& Gallagher, P.~T.\\
``Density and Alfv\'{e}n....'', \\
\emph{Some Journal}, Volume X, Issue Y, article id. (2013)

\item Bloomfield, S.~D., \textbf{Carley, E.~P.},\\
``A Comprehensive Overview of the 2011 June 7 Solar Storm'', \\
\emph{Astronomy \& Astrophysics}, Volume X, Issue Y, article id. (2013)


\end{enumerate}

