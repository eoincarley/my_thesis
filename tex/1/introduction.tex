%!TEX root = ../thesis.tex
%Adding the above line, with the name of your base .tex file (in this case "thesis.tex") will allow you to compile the whole thesis even when working inside one of the chapter tex files
%: ----------------------- introduction file header -----------------------
\chapter{Introduction}
\label{chap:1}

The Sun has long been the focus of humanity's curiosity. Throughout history it has been the harbinger of new religions, philosophies, and sciences. It has changed our understanding of our place in the Universe and allowed us to push forward the frontiers of stellar astronomy. Although our understanding of the Sun is nowadays more advanced, the curiosity we hold for it has not changed since the very early humans.
Now, we understand the Sun is a star similar to any other in its class, currently going through a relatively unchanging 11 year cycle of activity that is extremely rich in physical complexity. The study of such complex phenomena has yielded immeasurable advances in many areas of physics such as spectroscopy, plasma physics, magnetohydrodynamics (MHD), particle physics, to name but a few. Although some of these sciences have grown over decades (or even centuries) they are still incomplete. I hope this theses, in some small way, will contribute to the continuing growth of these sciences and to the understanding of our nearest star.


%Here is the introduction of the thesis, complete with a few references  \citep{sagan1997demon, prothero2007evolution}.  Section \ref{sec:1} contains Equation \ref{eqn:1}, Section \ref{sec:2} has Figure \ref{fig:1} and Section \ref{sec:3} has Table \ref{tab:1}. Chapter \ref{chap:2} has pretty much nothing in it.

\section{The Sun}\label{sec:1}

The Sun is our nearest star, located $1.49\times10^6$\,km from Earth at the centre of our solar system. Located on the main sequence of the Hetzpring-Russel (HR) diagram, it has a spectral class of G2V, with a luminosity of $L_{\odot}=(3.84\pm 0.04)\times10^{26}$\,W, mass of $M_{\odot}=(1.989\pm0.0003)\times10^{30}$\,kg and radius of $R_{\odot}=(6.959\pm0.007)\times10^8$\,m \citep{foukal2004}. It was born approximately $4.6 \times 10^9$\,years ago when a giant molecular cloud underwent gravitational collapse and began hydrogen nuclear fusion at its centre (reference). The energy produced from this fusion resulted in enough pressure to counteract gravitational contraction and bring about a hydrostatic equilibrium, allowing the young star to reach a stability that is sustained today. It is estimated the Sun will maintain this stability for another 5 billion years, at which point, it will move off the main sequence and into a red giant phase. During this later part of its life, it will grow in size to 100 times its current radius and begin nuclear burning of heavier elements such as carbon and oxygen. Once carbon burning in the core has ceased it can no longer sustain nuclear fusion of heavier elements, resulting in a gravitational instability that will eventually lead to a stellar nova. This nova will result in the loss of the outer envelopes and ultimately the Sun's death, leaving behind a compact and dense white-dwarf.

Until such time, the Sun will remain on the main sequence in a regular state of hydrogen fusion in its core. The energy released during this process is the ultimate source of light and all energetic activity that we observe from Earth and beyond. Before we can understand how this energy manifests in the solar atmosphere as a variety of energetic phenomena, it is important to understand how the energy is generated and transported through its interior and finally released into its atmosphere and interplanetary space.

\subsection{Solar Interior}\label{sec:10}

The theoretical development on how the solar interior is structured and how it behaves has been through what is known as the \textquoteleft standard solar model' or SSM. The SSM is a grouping of theories that described how the Sun was formed, how it maintains its stability, how it generates energy, and how this energy is transported through its interior and released at the surface. Much of the major developments of this theory have been in the 20th century, due mainly to the pioneering experiments in solar neutrino physics and helioseismsology. Hence, the development of the SSM has mainly been through a refinement of the theory based on these observational fields. Although the SSM has increased in sophistication, its four main aspects remain the most general framework for describing the behavior of the solar interior.

The SSM firstly states that the Sun was born from the gravitational collapse of a primordial gas of hydrogen, helium, and traces of other heavy elements. Secondly, it maintains its structural stability via a hydrostatic equilibrium such that the gravitational force is balanced by a pressure gradient ($\grad{P}=-\rho g$) at each radial distance inside the star. The third main aspect of the SSM involves the source of the Sun's energy. Much of the early ideas proposed during the 19th century involved some form of chemical reaction or energy released during a slow gravitational contraction. However, during the first half of the 20th century the theory that the Sun is at least as old as the Earth began to come into focus. The idea of the Sun being more than 4.5 billion years old prompted the question of what energy source could sustain the Sun's luminosity for such a length of time. It was soon realised that thermonuclear fusion must be the source of such energy, and, as a result, it should be possible to observe the neutrino products of this fusion. Hence, starting in the 1950s a number of pioneering neutrino physics experiments were developed in an attempt to detect solar-generated neutrinos at Earth. These pioneering experiments, as well as there more sophisticated counterparts today, confirm much of the theories on solar core energy generation.

From the 1950s onwards there has been a confirmed detection of neutrinos generated in a hydrogen fusion process, namely the proton-proton or \textquoteleft pp'-chain, in the solar core. In this process, four protons are fused to form a helium nucleus. This can occur in a variety of ways, but at the Sun's core temperature of 15 MK, the dominant reaction is the pp 1 chain given by
\begin{equation}
^{1}_1\mathrm{H} + ^{1}_1\mathrm{H} \rightarrow ~^{2}_1\mathrm{H} + e^{+}  + \nu_e
\end{equation}
\begin{equation}
^{2}_1\mathrm{H} + ^{1}_1\mathrm{H} \rightarrow ~^{3}_2\mathrm{He} + \gamma
\end{equation}
\begin{equation}
^{3}_2\mathrm{He}+^{3}_2\mathrm{He} \rightarrow ~^{4}_2\mathrm{He} + 2^{1}_1\mathrm{H}
\end{equation}
where $^{1}_1\mathrm{H}$ is a hydrogen nucelus, $^{2}_1\mathrm{H}$ is deuterium, $^{3}_2\mathrm{He}$ is tritium, $^{4}_2\mathrm{He}$ is helium, $e^{+}$ is a positron, $\nu_e$ is an electron neutrino, and $\gamma$ is a gamma ray photon. Reactions (1.1) and (1.2) must happen twice for (1.3) to occur. Taking this into account, the entire process may be summarised as 
\begin{equation}
4 ^{1}_1\mathrm{H}  \rightarrow ~^{4}_2\mathrm{He} + 2e^{+} + 2\nu_e + 2\gamma
\end{equation}
liberating $4.2\times10^{-12}$J of energy, with $\sim2.4$\% of the energy carried away by the neutrinos. This particular form of the pp-chain (pp 1) occurs in 86\% of the cases \citep{turk2011}. However, there are other reactions capable of producing He from H catagorized into pp II, pp III etc, which each involve production of $^7_4$Be and $^8_5$B. The initial neutrino detections at Earth were the result of the pp III reaction which involves the creation of $^8_5$B, followed by a decay to $^8_4$Be, a positron, and an electron neutrino \citep{davis1968}. These early detections and the results of more recent experiments such as the SuperKamiokande \citep{fukuda1998} show that the expected neutrino flux given by the standard solar model is smaller than the observed. This deficit in in neutrino flux observations became the famed \textquoteleft solar neutrino problem' during the 1970s. 
One of the proposed explanations for the process was via an oscillation of the neutrino amongst three sets of 'flavors' i.e., the neutrino can be either an electron $\nu_e$, muon $\nu_{\mu}$, or tau $\nu_{\tau}$ neutrino. With the original detectors only being able to detect the $\nu_e$, this would result in a flux deficit (non-detection of $\nu_{\mu}$ and $\nu_{\tau}$). This oscillation amongst three flavors was given the name the \textquoteleft MSW effect' after \citet{mikheev1986} and \citet{wolfenstein1978}, and later confirmed experimentally by the SuperKamionkande experiment.

The neutrino experiments together with the standard solar model SSM provide much of what we know about the solar energy generation and the solar core. They imply a temperature of $15.6\times10^6$K and density of $1.48\times10^5$\,kg\,m$^{-3}$ at solar centre, and also confirm the existence of a variety of pp reactions (pp 1 to pp IV), and some level of Carbon-Nitrogen-Oxygen (CNO) fusion process. These fusion processes occur over $0.0-0.25\,R_{\odot}$ (Figure~\ref{fig:solar_atmosphere}), which defines the solar core. Outside the core the temperature drops to a value such that fusion ceases. While thermonuclear fusion is the third aspect of the SSM involves the generation of solar energy, the fourth aspect involves exactly what happens to this energy once it is generated i.e., it describes an energy transport mechanism.


Beyond $0.25\,R_{\odot}$ the temperature drops to 8 MK, such that fusion stops but only free protons and electrons exist. In this environment, the photons continuously scatter off free particles, undergoing a random walk toward the surface over a distance of $0.25-0.7\,R_{\odot}$. This region is known as the radiative zone and has densities of $2\times10^4-2\times10^2$\,kg\,m$^{-3}$, resulting in a small photon mean free path (mfp) of $9.0\times10^{-4}$\,m. The photons proceed towards the solar surface over a very long time scale, taking on the order of $10^{5}$ years to traverse this region \citep{mitalas1992}. If radiative energy transport occurs, it will result in the following temperature gradient
\begin{equation}
\frac{dT}{dr} = -\frac{3}{16 \sigma}\frac{\kappa \rho}{T^3}F_{rad}
\end{equation}
where $\sigma$ is the Stefan-Boltzman constant, $\kappa$ is the mass extinction coefficient (opacity per unit mass), $\rho$ is mass density, $T$ is temperature, and $F_{rad}$ is the outward radiative flux. This implies that for a particular outward flux, if the opacity increases, a steeper temperature gradient is required to maintain such a flux. At $0.7\,R_{\odot}$ the temperature drops to 0.5\,MK allowing protons to capture electrons into a bound orbit. The existence of electrons in atomic orbit results in a dramatic increase in opacity of the plasma \citep{turk2011} and hence the temperature gradient increases. The increased temperature gradient required to sustain the energy flow may lead to the onset of a convective instability beyond $0.7\,R_{\odot}$ toward the solar surface. This instability will occur if the temperature gradient in the star is steeper than the adiabatic temperature gradient
\begin{equation}
\Bigg|\frac{dT}{dr}\Bigg|_{star} > \Bigg|\frac{dT}{dr}\Bigg|_{adiabatic}
\end{equation}
This is known as the Schwarzchild criterion, and it is fulfilled from $0.7-1\,R_{\odot}$ $-$a region known as the convection zone. The temperature and density drop as height increases and finally reaches T$\sim$$6000$\,K and mass densities of $\rho\sim1\times10^{-5}$\,kg\,m$^{-3}$. Although no complete theoretical treatment of convection exists, mixing length theory and hydrodynamical modeling are used to determine how convection occurs in the solar interior.

Much of what we know about the depth, temperature, and density of the core, radiative, and convection zones come from a fine-tuning of the standard solar model, such that the model reproduces observations from neutrino and helioseismology experiments. In fact helioseismology alone can indicate great detail of the internal structure of the Sun. It has revealed that both the core and radiative zone rotate as a rigid body, while the convective zone undergoes differential rotation (REFERENCE), in much the same way as the solar surface does. 
Hence the boundary between the radiative and convective zones mark a region where the internal dynamics of the Sun change dramatically. This boundary is known as the tachocline, and it is this region that is of much relevance to the generation and cyclic evolution of the Sun's magnetic field.


%The source of the Sun's energy is nuclear fusion in the solar core. Temperatures as high as $15\times10^{6}$\,K allow four protons to fuse and become a helium nucleus i.e., $4\,^{1}$H$\rightarrow ^{4}$He\,+\,2e$^{+}+2\nu+2\gamma$, in a process known as the proton-proton or pp-chain. Here e$^+$, $\nu$, and $\gamma$ are a positron, neutrino, and gamma ray photon, respectively, resulting from fusion processes in the pp-chain. The solar core extends to approximately $0.25\,R_{\odot}$ from solar center where hydrogen burning (fusion) stops. Beyond this point, energy transport is dominated by photons scattering off of free particles. The transport of energy via radiation continues up to $\sim0.8\,R_{\odot}$, at which point the temperature is low enough such that neutral atoms form and radiation can no longer propagate freely due to the high opacity. Between $\sim0.8-1\,R_{\odot}$ the temperature gradient is large enough for convection to become the dominant mechanism for the transport of energy to the solar surface


\begin{figure}[!h]
\begin{center}
\includegraphics[trim = 0cm 0.5cm 0cm 0cm, width=1.0\textwidth]{images/solar_atmosphere}
\caption{The internal structure of the Sun, including the core, radiative zone, and convective zone.  Also shown is the structure of the its atmosphere, including the photosphere, chromosphere, and corona. The layers of the solar atmosphere are usually demarcated by temperature changes as height above the solar surface increases. The temperature ranges from $\sim$6000\,K in the photosphere to above 1\,MK in the corona.}
\label{fig:solar_atmosphere} 
\end{center}
\end{figure}



\subsection{Solar Dynamo and Magnetic Field}\label{sec:11}

\begin{itemize}
\item Cellular pattern in photosphere is very different to the convective structure in deeper layers.
\end{itemize}


\begin{figure}[!h]
\begin{center}
\includegraphics[]{images/Babcock}
\caption{Differential rotation and flux freezing result in the poloidal dipolar magnetic field, generated by dynamo action, to be dragged around in a toroidal direction, an action known as the omega effect. Buoyancy of the field lines results in them rising and twisting, known as the alpha effect, eventually surfacing to become bipolar fields that extend far into the corona.}
\label{fig:Babcock} 
\end{center}
\end{figure}

It is widely believed that the Sun's magnetic field is created by a dynamo action in a region between the radiative zone and the convection zone, known as the tachocline. Solar dynamo theory attempts to explain the observed 11 year magnetic activity cycle, where the the Sun's magnetic field starts as a poloidal dipolar structure and evolves to having a strong toroidal component, after which it returns to a poloidal field again. During these 11 years the Sun starts at minimum activity, reaches a maximum and returns to minimum again.

\citet{babcock1961} first explained this process by a mechanism involving differential rotation of the solar surface and interior. The equatorial rotation rate is faster than the rotation rate at higher latitudes. Because the magnetic field is frozen into the plasma, any flows in the solar interior will tend to drag the magnetic field along. By this effect, differential rotation tends to drag the field and wrap it around the Sun in a toroidal direction, this is known as the omega effect, see Figure~\ref{fig:Babcock}.

As the toroidal field builds up in the solar interior, sections of field lines build up in magnetic pressure resulting in a buoyancy of the field. The field slowly rises through the convection zone and eventually surfaces as a bipolar region that extends into the solar atmosphere. The presence of bipolar fields in the solar atmosphere and their slow build up over time to complex magnetic structures, known as active regions, ultimately leads to a variety of eruptive phenomena.



\subsection{Solar Atmosphere}\label{sec:12}

\subsubsection{Photosphere}\label{sec:121}

\begin{itemize}
\item Appearance, Granules, Sunspots.
\item Black Body Curve. Franhofer lines, H-alpha line, CaII H \& K, H$^{-}$ alines, Sodium D lines. 
\item Temperature, Density, Opacity.
\item Magnetic field strength
\end{itemize}

The most spectacular and energetic phenomena in our solar system have their origins in the solar atmosphere. This ever-changing and dynamic environment is a hotbed of activity giving rise to coronal mass ejections (CMEs), solar flares, and a host of plasma processes resulting in emission across the entire electromagnetic spectrum. To make sense of the phenomena we observe we must first have a basic understanding of solar atmospheric structure and the environment these processes take place in. Figure~\ref{fig:solar_atmosphere} is an illustration of the different layers of the solar interior, the solar surface and atmosphere. The visible surface of the Sun is known as the photosphere. It is demarcated where optical depth becomes unity for a wavelength of 5000\,\AA\ or $\tau_{5000}=1$. At such visible wavelengths, the electromagnetic spectrum is well represented by a blackbody of temperature T$\sim$6000\,K. .

\begin{itemize}
\item Eddington Barbier, $\tau=\mu$, limb darkening.
\item Effective temperature, $\tau=2/3$, $T=5800$\,K
\end{itemize}

During periods of increased activity there may also be the presence of sunspots in the photosphere. These are dark features on the solar surface, see Figure~\ref{fig:solar_atmosphere}, and are an indicator of concentrations of magnetic fields that are stronger than elsewhere in the quiet sun, as described above. 

Photospheric abundances have been measured using emission line diagnostics were it is found that helium is the most abundant at 10.89\footnote{Abundances quoted relative to hydrogen on a logarithmic scale, $12.0+log_{10}(A/A_H)$}, with the next most abundant elements being Carbon (8.58), Nitrogen (8.02), and Oxygen (8.8). All other elements have abundances that are 4 orders of magnitude or more less than hydrogen i.e., logarithmic abundances $\lesssim7$ \citep{phillips2008}.

\subsubsection{Chromosphere}\label{sec:122}

\begin{itemize}
\item Appearance, Supergranular Network, Bright Points, Spicules, Filaments, Plage etc.
\item Emission lines, H-alpha, CaII H \& K. 
\item Temperature, Density, Opacity.
\item Magnetic field strength.
\end{itemize}

At $\sim$500\,km above the $\tau_{5000}=1$ surface the temperature drops to a minimum of $\sim$4400\,K. Beyond this minimum the temperature begins to rise again, demarcating the beginning of the chromosphere. This layer of the atmosphere is generally accepted to extend to a height at which temperatures reach 20,000\,K, however temperatures as high as $\sim1\times10^5$\,K are sometimes attributed to chromospheric heights, hence it is observable at ultraviolet (UV) wavelengths as well as visible. 

\subsubsection{Corona}\label{sec:123}

\begin{itemize}
\item Appearance UV: Active regions, Coronal Loops, Holes.
\item Emission lines, Mg, Ca, Fe, C, O etc.
\item Appearance White-Light: Streamers, K, F, E corona
\item Appearance Radio: thermal bremsstrahlung, free-free emissivity/opacity.
\item Temperature, Density, Opacity, 'coronal heating problem'.
\end{itemize}


At a height of approximately 2,000\,km the temperature begins to rise sharply while the number density of neutral hydrogen and electrons fall by several orders of magnitude. This rapid increase in temperature in such a short spatial extent ($<$100\,km) is known as the transition region. It has a temperature on the order of $10^5$\,K and separates the relatively low temperature chromosphere and the high temperatures of $>1$\,MK in the corona. The reason for this rapid increase in temperature is still a hotly debated subject and a coronal heating mechanism remains largely unknown, this is known as the  \textquoteleft coronal heating problem'.


Element abundances in the corona show there is a similar composition to the photospheric abundances, with He, C, N, and O having the same ratios relative to H in the corona as that in the photosphere. The only difference is an enhancement in the abundance of low First Ionization Potential ($<10$\,eV) elements in the corona relative to the photosphere. For example, elements such as Na, Mg, Al, Si, Ca, Ni, and Fe can be up to three times more abundant in the corona \citep{feldman2003}. The reason for the enhancement of low FIP elements in the corona is still unknown, however several models have suggested ion-neutral separation in the chromosphere by diffusion across magnetic fields, followed by transport of these ions into the corona may be viable mechanism \citep{geiss1985}. 


\subsection{Solar Wind}\label{sec:13}

\begin{itemize}
\item Parker's solution
\item Parker Spiral
\item Fast solar wind, Alfv\'{e}n wave driver
\item Mass loss rates (later compare CME mass loss)
\end{itemize}



\section{Coronal Mass Ejections and Coronal Shocks}\label{sec:2}

\subsection{CMEs}\label{sec:20}

\begin{itemize}
\item Appearance, white-light Illing, Hundhausen, Vourlidas
\item Kinematics, velocity, acceleration
\item Dynamics, masses, energies, forces
\item Observations at other wavelengths, EUV, radio, SXR.
\end{itemize}

\subsection{CMEs and Shocks}\label{sec:21}

\begin{itemize}
\item Radio bursts, Type II, Type III
\item Radio imaging of shocks
\item Relationship to EUV waves, Moreton waves
\end{itemize}

\subsection{Open Questions}\label{sec:22}






