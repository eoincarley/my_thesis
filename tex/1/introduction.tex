%!TEX root = ../thesis.tex
%Adding the above line, with the name of your base .tex file (in this case "thesis.tex") will allow you to compile the whole thesis even when working inside one of the chapter tex files
%: ----------------------- introduction file header -----------------------
\chapter{Introduction}
\label{chap:1}

The Sun has long been the focus of humanity's curiosity. Throughout history it has been the harbinger of new religions, philosophies, and sciences. It has changed our understanding of our place in the Universe and allowed us to push forward the frontiers of physical science. Although our understanding of the Sun is nowadays more advanced, the curiosity we hold for it has not changed since the very early humans.
Now, we understand the Sun is a star similar to any other in its class, currently going through a relatively unchanging 11 year cycle of activity from solar minimum to solar maximum and back again. Most of the phenomena described here result mainly from the energetic processes occurring during solar activity maximum.


%Here is the introduction of the thesis, complete with a few references  \citep{sagan1997demon, prothero2007evolution}.  Section \ref{sec:1} contains Equation \ref{eqn:1}, Section \ref{sec:2} has Figure \ref{fig:1} and Section \ref{sec:3} has Table \ref{tab:1}. Chapter \ref{chap:2} has pretty much nothing in it.

\section{The Sun}\label{sec:1}


\subsection{Solar Interior}\label{sec:10}

\subsection{Solar Dynamo and Magnetic Field}\label{sec:11}

\subsection{Solar Atmosphere}\label{sec:12}

\subsection{Solar Wind}\label{sec:13}





\section{Coronal Mass Ejections and Coronal Shocks}\label{sec:2}

\subsection{Observations}\label{sec:20}

\subsection{Current understanding}\label{sec:21}

\subsection{Open Questions}\label{sec:22}






