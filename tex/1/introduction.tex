%!TEX root = ../thesis.tex
%Adding the above line, with the name of your base .tex file (in this case "thesis.tex") will allow you to compile the whole thesis even when working inside one of the chapter tex files
%: ----------------------- introduction file header -----------------------
\chapter{Introduction}
\label{chap:1}

The Sun has long been the focus of humanity's curiosity. Throughout history it has been the harbinger of new religions, philosophies, and sciences. It has changed our understanding of our place in the Universe and allowed us to push forward the frontiers of stellar astronomy. Although our understanding of the Sun is nowadays more advanced, the curiosity we hold for it has not changed since the very early humans.
Now, we understand the Sun is a star similar to any other in its class, currently going through a relatively unchanging 11 year cycle of activity that is extremely rich in physical complexity. The study of such complex phenomena has yielded immeasurable advances in many areas of physics such as spectroscopy, plasma physics, magnetohydrodynamics (MHD), particle physics, to name but a few. Although some of these sciences have grown over decades (or even centuries) they are still incomplete. I hope this theses, in some small way, will contribute to the continuing growth of these sciences and to the understanding of our nearest star.


%Here is the introduction of the thesis, complete with a few references  \citep{sagan1997demon, prothero2007evolution}.  Section \ref{sec:1} contains Equation \ref{eqn:1}, Section \ref{sec:2} has Figure \ref{fig:1} and Section \ref{sec:3} has Table \ref{tab:1}. Chapter \ref{chap:2} has pretty much nothing in it.

\section{The Sun}\label{sec:1}

The Sun is our nearest star, located $1.49\times10^6$\,km from Earth at the centre of our solar system. It is main sequence star of spectral class G2V, with a luminosity of $L_{\odot}=(3.84\pm 0.04)\times10^{26}$\,W, mass of $M_{\odot}=1.989\times10^{30}$\,kg and radius of $R_{\odot}=6.985\times10^8$\,m. It was born approximately $4.6 \times 10^9$\,years ago when a giant molecular cloud underwent gravitational collapse and began hydrogen nuclear fusion at its centre (reference). The energy produced from this fusion resulted in enough pressure to counteract gravity and bring about a hydrostatic equilibrium ($\nabla P = -\rho g$) allowing the young star to reach a stability that is sustained today. It is estimated the Sun will maintain this stability for another 5 billion years, at which point, it will move off the main sequence and into the red giant phase. During this later part of its life, it will grow in size to 100 times its current radius and begin nuclear burning of heavier elements such as carbon and oxygen. Once carbon burning in the core is depleted it can no longer sustain nuclear fusion of heavier elements, resulting in a gravitational instability that will eventually lead to a stellar nova. This nova will result in the loss of the outer envelopes and ultimately the Sun's death, leaving behind a compact, dense white-dwarf.

Until such time, the Sun will remain on the main sequence in a regular state of hydrogen fusion in its core. The energy released during this process is the ultimate source of light and all energetic activity that we observe from Earth and beyond. Before we can understand how this energy manifests in the solar atmosphere as a variety of energetic phenomena, it is important to understand how the energy is generated inside the star and how it is transported through its interior.

\subsection{Solar Interior}\label{sec:10}

The source of the Sun's energy is nuclear fusion in the solar core. Temperatures as high as $15\times10^{6}$\,K allow four protons to fuse and become a helium nucleus i.e., $4\,^{1}$H$\rightarrow ^{4}$He\,+\,2e$^{+}+2\nu+2\gamma$, in a process known as the proton-proton or pp-chain. Here e$^+$, $\nu$, and $\gamma$ are a positron, neutrino, and gamma ray photon, respectively, resulting from fusion processes in the pp-chain. The solar core extends to approximately $0.25\,R_{\odot}$ from solar center where hydrogen burning (fusion) stops. Beyond this point, energy transport is dominated by photons scattering off of free particles. The transport of energy via radiation continues up to $\sim0.8\,R_{\odot}$, at which point the temperature is low enough such that neutral atoms form and radiation can no longer propagate freely due to the high opacity. Between $\sim0.8-1\,R_{\odot}$ the temperature gradient is large enough for convection to become the dominant mechanism for the transport of energy to the solar surface

\subsection{Solar Dynamo and Magnetic Field}\label{sec:11}

\subsection{Solar Atmosphere}\label{sec:12}

\subsection{Solar Wind}\label{sec:13}





\section{Coronal Mass Ejections and Coronal Shocks}\label{sec:2}

\subsection{Observations}\label{sec:20}

\subsection{Current understanding}\label{sec:21}

\subsection{Open Questions}\label{sec:22}






