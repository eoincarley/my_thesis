%!TEX root = ../thesis.tex
%Adding the above line, with the name of your base .tex file (in this case "thesis.tex") will allow you to compile the whole thesis even when working inside one of the chapter tex files
\chapter{Coronal Mass Ejection and Plasma Shock Theory} 
\label{chap:2}

\section{Plasma Physics and Magnetohydrodynamics}\label{sec:1}

\subsection{Maxwell's Equations}\label{sec:10}

\begin{eqnarray}
\nabla \cdot E &=& \frac{\rho}{\epsilon_0} \\
\nabla \cdot B &=& 0 \\
\curl{E} &=& - \frac{dB}{dt} \\
\curl{B} &=& \mu_0 j + \frac{1}{c^2}\frac{dE}{dt} 
\end{eqnarray}



\subsection{Plasma Physics and Boltzmann Equation}\label{sec:11}

\subsection{Magnetohydrodynamics}\label{sec:12}

\subsection{Magnetic Reconnection}\label{sec:13}

\subsection{MHD Rankine-Hugoniot Equations}\label{sec:14}

\subsection{Bow Shocks}\label{sec:15}



\section{Coronal Mass Ejections}\label{sec:2}

\subsection{Catastrophe Model}\label{sec:20}

\subsection{Magnetic Breakout Model}\label{sec:21}

The magnetic breakout model was first proposed by \citep{antiochos1999} and involves a quadrupolar (or more complex) magnetic flux system. A core magnetic flux is flanked by two side-lobe flux systems. The core and two side lobes lie underneath a field that provides a tension, stabilizing the entire system. Above the core flux there is a magnetic null point. Non potentiality is injected into the core flux by twisting/shearing of the magnetic footpoints or flux emergence. This non-potentiality causes the core flux to grow and encounter the overlying field creating the tension. Contact of the core flux and overlying flux creates a current sheet above the core, allowing reconnection to occur. This reconnection removes flux from the overlying field and adds it to the side-lobe flux systems, allowing the core flux to escape. The magnetic breakout model was used to circumvent the Aly-Sturrock limit (reference) i.e., it allowed a flux system to erupt, without having to open the constraining field lines to infinity.
\begin{figure}[!t]
\begin{center}
\includegraphics[scale=0.5, trim=1cm 2cm 1cm 1cm]{images/Antiochos_breakout}
\caption{The breakout model, consisting of a quadrupolar flux system in which the central flux (blue) is flanked by to side lobe flux systems (green), with the entire system kept in stability by the tension of the overlying red field. Shearing and/or twisting on the underlying flux causes it to grow slowly. Eventually a current sheet forms at the magnetic null above the central flux, causing reconnection. This reconnection transfers overlying field to the side-lobes, effectively creating a conduit for the central flux to escape as a CME.}
\label{fig:breakout_model}
\end{center}
\end{figure}
There has been observational tests of the magnetic-breakout model, showing it to be a viable explanation of some flaring and CME events. The most notable of these is the Bastille Day event (reference Amarai).The obseronal signatures of the model include the presence of a null point in the corona above a complex quadrupolar flux system (inferred from potential field source surface extrapolations), a radio source imaged to be above the erupting structure (implying a reconnection site), and radio bursts beginning at frequencies indicative of high altitude (again indicating energy release above the erupting structure, prior to eruption) - reference Manoharan. However, in some instances magnetic breakout is implied by observations of the above, but the kinematics are inconsistent with model predictions. For example the model predicts a long slow rise of the central flux system as the underlying field is increasingly sheared. The CME/central flux begins rapid acceleration once breakout reconnection is initiated. However, in the study of \citet{bong2006} the breakout reconnection occurred at the end of the CME acceleration phase, prompting a two-phase acceleration scenario.

Kinematcally the CME/central flux system should experience a slow rise (1 km/s) for several hours due to shearing/twisting of the footpoints. Once breakout reconnection has begun the CME experiences a much larger acceleration (100 km/s) for $\sim$ 1 hour. The transference of magnetic flux to the side lobes drives the flare and is the source of energetic particles that lead to flare hard X-rays and ultimately soft X-rays. Therefore magnetic breakout predicts that the flaring process and SXR peak should only begin after CME acceleration (after breakout reconnection) \citep{lynch2004}. However, the precedence is breakout reconnection over flaring reconnection may not always be case, with latter sometimes driving the former \citep{macneice2004}

\subsection{Toroidal Instability}\label{sec:22}

The toroidal instability model incorporates a pre-existing flux rope structure that is built from a torus of magnetic flux, some of which is buried beneath the photosphere \citep{chen1989}. The flux system is can be broken down into a combination of toroidal magnetic, toroidal current and a poloidal magnetic field and current Figure~\ref{fig:chen_model}. This flux rope system is embedded in a surrounding coronal magnetic field $B_{corona}$. The stability of the system depends on the nature of the $J \times B$ force due to the interaction toroidal and poloidal components of both the field and current. The interaction of $J$ and $B$ internal to the flux rope is usually termed the Lorentz self-force or the \textquoteleft hoop' force. An instability may be induced via twisting of the fluxrope footpoints to increases the amount of poloidal flux (effectively increasing the helicity of the system). Once the instability is induced, the fluxrope begins a bulk motion as well as a growth in its semi-minor axis. Hence the motion of the system can be analysed by looking at the central axis or the minor axes (leading and trailing edges. The three axes display slightly different kinematics e.g., the leading edge has a faster velocity than the trailing edge (due to fluxrope expansion). this has proved a useful test of the model when comparing the observations of erupting fluxrope structures as seen in white-light coronagraphs. \citet{krall2001} looked at the leading a trailing edges of erupting flux ropes, as well as the rope aspect ratio, an compared the observations to model expectations. Good agreement is found between the model kinematics and aspect ratio and the observed events.
\begin{figure}[!t]
\begin{center}
\includegraphics[scale=0.5]{images/chen_model}
\caption{The flux rope model of \citet{chen1989}, used to to study the toroidal instability of a twisted flux system in the corona.}
\label{fig:chen_model}
\end{center}
\end{figure}
The equation of motion of the entire system is given by
\begin{equation}
M\frac{d^2Z}{dt^2} = \frac{I_t}{c^2R}\times\bigg[ \mathrm{ln}\bigg(\frac{8R}{a}\bigg) -1+ \frac{\xi_i}{2} + \frac{\beta_p}{2} -\frac{B^2_t}{B^2_{pa}}  -\frac{2RB_{\perp c}}{aB_{pa}} \bigg] - F_g - F_{drag}
\end{equation}
where $I_t$ is the toroidal current, $R$ is the flux rope major radius, $a$ is the rope minor radius, $\xi_i$ is internal inductance of the flux system, $B_t$ is the toroidal field, $B_{pa}$ is the poloidal field at $a$, $B_{\perp c}$ is the perpendicular component of the ambient coronal field, $F_g$ is the force due to gravity, $F_{drag}$ is the drag force, $M$ is the mass per unit length of the rope, and $Z$ is the rope axis height above the photosphere. The equation of motion shows that an increase in the toroidal current (or poloidal flux) contributes positively to the acceleration. The terms in the square brackets are each unitless and take into account the rope geometry, self-inductance and interplay between poloidal and toroidal flux. The first three terms in the square brackets are what give rise to the hoop-force. If the rope is mass loaded with a prominence, this can contribute to the rope's stability via the gravity term. The drag term only becomes an important contributor to rope dynamics later in the propagation, when the solar wind speed begins to increase i.e., at around 10$R_{\odot}$ reference Sheeley. The eruption is driven by flux-injection, which typically lasts for 4-8 hours, during which time the unstable system loses its equilibrium and begins to rise \citet{krall2001}.

It is significant the fluxrope is already established in the corona before eruption begins i.e., the rope formation is not addressed in the model and it is not a consequence of eruption. Hence magnetic reconnection is not a necessary aspect of the model and the eruption may proceed without employing resistive MHD.


\subsection{Drag Models}\label{sec:23}


\section{Coronal Shocks and Plasma Emission}\label{sec:3}

\subsection{Shock Particle Acceleration}\label{sec:30}

\subsection{Wave-Particle Interaction}\label{sec:31}

Quasi-linear relaxation
\begin{figure}[!t]
\begin{center}
\includegraphics[scale=0.35, trim = 4cm 0cm 0cm 0cm]{images/Grognard1975}
\end{center}
\end{figure}


\subsection{Three-Wave Interaction and Plasma Emission}\label{sec:32}

Once the Langmuir waves are produced from the bump-on-tail instability a number of wave interaction processes occur in order to bring about plasma emission. This involves the interaction of various wave modes in the plasma described by a mathematical formalism called the three-wave interaction. In this process three wave modes in a plasma M, P, and Q are described by their distribution functions in a wave-number space ($k$-space). the distribution functions are given by $N_M(k_M)$, $N_P(k_P)$, $N_Q(k_Q)$, where the $N$ describe the occupation number of wave quanta between $k$ and $k+dk$ in the wave-number space. Waves in P and Q mode may interact to such that wave quanta are removed from the P and Q k-space and added to the M k-space. This is essentially an emission of an energy packet from the P and Q -space to the M k-space. The rate of change of occupation numbers in the three k-spaces are given by
\begin{equation}
\frac{dN_M(\mathbf{k}_M)}{dt} = -\int \frac{d^3\mathbf{k}_P}{(2\pi)^3}\int \frac{d^3\mathbf{k}_Q}{(2\pi)^3}g(\mathbf{k}_M, \mathbf{k}_P, \mathbf{k}_Q)
\end{equation}
\begin{equation}
\frac{dN_P(\mathbf{k}_P)}{dt} = -\int \frac{d^3\mathbf{k}_M}{(2\pi)^3}\int \frac{d^3\mathbf{k}_Q}{(2\pi)^3}g(\mathbf{k}_M, \mathbf{k}_P, \mathbf{k}_Q)
\end{equation}
\begin{equation}
\frac{dN_Q(\mathbf{k}_Q)}{dt} = -\int \frac{d^3\mathbf{k}_M}{(2\pi)^3}\int \frac{d^3\mathbf{k}_P}{(2\pi)^3}g(\mathbf{k}_M, \mathbf{k}_P, \mathbf{k}_Q)
\end{equation}
where $g(\bf{k}_M, \bf{k}_P, \bf{k}_Q)$ is incorporates a transition probability for wave quanta into and out of energy states in the various k-spaces \citep{robinson1994}. The transition probability is given by
\begin{multline}
g(\mathbf{k}_M, \mathbf{k}_P, \mathbf{k}_Q) = u_{MPQ}(\mathbf{k}_M, \mathbf{k}_P, \mathbf{k}_Q)   [N_M(\mathbf{k_M}) N_P(\mathbf{k_P})  - \\ 
N_P(\mathbf{k_P}) N_Q(\mathbf{k_Q})   +N_Q(\mathbf{k_Q}) N_M(\mathbf{k_M})  ]
\end{multline}
where $u_{MPQ}(\bf{k}_M, \bf{k}_P, \bf{k}_Q) $ is the transition probability from states in P and Q  to M, for example \citep{melrose1986}. $u_{MPQ}$ is analogous to transition probabilities given by the Einstein coefficients for transferring energy packets from and atomic state to a photon state (photon emission) i.e., whereas the Einstein coefficients are used in atom-wave (atom-photon) energy exchanges, $u_{MPQ}$ describes wave-wave energy exchanges. The transition probability is given by
\begin{equation}
u_{MPQ}(\mathbf{k}_M, \mathbf{k}_P, \mathbf{k}_Q)  \propto \delta(\omega_M - \omega_P - \omega_Q ) \delta^3(\mathbf{k}_M - \mathbf{k}_P - \mathbf{k}_Q )
\end{equation}
where the $\omega$ are the frequency of the corresponding wave and and $\delta$ are delta functions. Given the presence of delta functions in the transition probability expression, we can see that an exchange of energy quanta amongst the wave modes can only occur when 
\begin{eqnarray}
\omega_M & = & \omega_P + \omega_Q \\
\mathbf{k}_M & = & \mathbf{k}_P + \mathbf{k}_Q
\end{eqnarray}

Hence for an a conversion wave modes in a plasma such as $M \rightarrow P + Q$ (a decay of mode M into P and Q), or it's reverse process $P + Q \rightarrow M $ (a coupling of P and Q to produce M) is described by equations (2.1) to (2.7). 

The production of plasma emission after a bump-on-tail instability has occurred requires a three wave interaction amongst a Langmuir wave $L$, ion acoustic wave $S$, and electromagnetic wave $T$. Fundamental emission during a radio burst occurs via a decay of Langmuir waves into an electromagnetic and ion sound wave
\begin{equation}
L \rightarrow T + S
\end{equation}
while second harmonic first requires the decay $L\rightarrow L^{'} + S$, where $L^{'}$ is a product Langmuir wave propagating in the opposite direction to the first. This is followed by a coalescence of the original and product Langmuir waves
\begin{equation}
L + L^{'}\rightarrow T'
\end{equation}


\begin{itemize}
\item The dispersion relations
\item Source emissivities
\end{itemize}


