%!TEX root = ../thesis.tex
%Adding the above line, with the name of your base .tex file (in this case "thesis.tex") will allow you to compile the whole thesis even when working inside one of the chapter tex files
\singlespacing
\chapter{Coronal Mass Ejection and Plasma Shock Theory} 
\label{chap:2}
\doublespacing
This chapter introduces the theory used to study coronal mass ejections and coronal shocks. Since the corona is a plasma, the theoretical framework under which all coronal phenomena are treated is in plasma physics and a fluid description of plasmas known as magnetohydrodynamics (MHD). Coronal mass ejections are a large scale phenomena and can therefore be treated using MHD. While plasma shocks on the large scale may also be treated in an MHD continuum framework, it is necessary to consider individual particle motions when describing particle acceleration and radio emission in shocks, requiring a departure from MHD and the use of distribution functions, the Boltzmann equation, and individual particle kinematics. Therefore, both the MHD equations and the Boltzmann equation are presented in this chapter, followed by an application of this theory to CMEs and plasma shocks.

\singlespacing
\section{Plasma Physics and  \\ Magnetohydrodynamics}\label{sec:1}
\doublespacing
\subsection{Maxwell's Equations}\label{sec:10}

Since plasmas are an electrically conducting fluid which may interact with electromagnetic fields, Maxwell's equations are an essential starting point for all plasma theory. Maxwell's equations form a closed set of four unknowns and four equations describing relationships between the electric field $\mathbf{E}$, the magnetic field $\mathbf{B}$, the current density $\mathbf{j}$, and the charge density $\rho_q$
\begin{eqnarray}
\nabla \cdot \mathbf{E} &=& \frac{\rho_q}{\epsilon_0} \\
\nabla \cdot \mathbf{B} &=& 0 \\
\curl{\mathbf{E}} &=& - \frac{d\mathbf{B}}{dt} \\
\curl{\mathbf{B}} &=& \mu_0 \mathbf{j} + \frac{1}{c^2}\frac{d\mathbf{E}}{dt} 
\end{eqnarray}
$\mu_0$ and $\epsilon_0$ are the magnetic permeability and electric permittivity of free space, respectively, and all bold face quantities represent are vector variables. At velocities typically found in a plasma equation (2.4), Amp\'{e}re's law, reduces to
\begin{equation}
\curl{\mathbf{B}} = \mu_0 \mathbf{j}
\label{eqn:amperes_law}
\end{equation}
where the displacement current is no longer included. Maxwell's equations describe electromagnetic behaviour and they constitute an important part of the fluid description of plasmas. Before we define this fluid description a brief discussion of plasma kinetic theory, from which the fluid theory is derived, is provided here.

\subsection{Plasma Kinetic Theory}\label{sec:11}

The general approach to the majority of plasma phenomenon is in a collective description using particle distribution functions and the use of differential equations to describe the evolution of these distribution functions. This is known as plasma kinetic theory, and the distribution functions can be of the form of the Maxwell-Boltzmann velocity distribution, while the differential equation used to describe its evolution is the Boltzmann equation. Many non-equilibrium or unstable states of a plasma, such as those that produce radio bursts described require a kinetic theory description of plasma. A brief overview of kinetic theory is therefore given here before describing specifically its application to plasma emission and radio bursts.

A particle distribution function $f(\mathbf{r}, \mathbf{v}, t)d\mathbf{v}d\mathbf{r}$ described the number of particles having positions between $\mathbf{r}$ and $\mathbf{r}+d\mathbf{r}$ and velocities between $\mathbf{v}$ and $\mathbf{v}+d\mathbf{v}$, at time $t$. This distribution function can be used derive a number of useful physical properties of the plasma, such as the particle number density at position $\mathbf{r}$ and time $t$ 
\begin{equation}
n(\mathbf{r},t) = \int f(\mathbf{r}, \mathbf{v},t) d \mathbf{v}
\label{eqn:num_density}
\end{equation}
as well as the bulk velocity particle velocity, given by
\begin{equation}
 \mathbf{v}(\mathbf{r},t) = \frac{1}{n(\mathbf{r},t)}\int  \mathbf{v} f(\mathbf{r}, \mathbf{v},t) d \mathbf{v}
\label{eqn:bulk_flow}
\end{equation}
The evolution of this distribution function in time and space is described by the Boltzmann equation, given by
\begin{equation}
\frac{\partial f(\mathbf{r}, \mathbf{v},t)}{\partial t}  +     ( \mathbf{v}\cdot\nabla_r)f(\mathbf{r}, \mathbf{v},t)    + (\mathbf{a}\cdot\nabla_v)f(\mathbf{r}, \mathbf{v},t) = \bigg(\frac{\partial f(\mathbf{r}, \mathbf{v},t)}{\partial t}\bigg)_{coll}
\label{eqn:boltzmann}
\end{equation}
This equation describes the changes in occupation number of particles at positions ($\mathbf{r}, \mathbf{v}$, t) in phase space due to a configuration space flux $\mathbf{v}\cdot\nabla_rf$, a velocity space flux $ \mathbf{a}\cdot\nabla_vf$, as well as collisions experienced by the particles (last term in the right). Equation~\ref{eqn:boltzmann} is the fundamental basis of all plasma and neutral gas kinetic theory and provides a very powerful tool for describing the time and space evolution of equilibrium and, more importantly, non-equilibrium distributions of particles. It is the most general description of the behavior of an ensemble of particles, and all other macroscopic fluid dynamical equations may be derived from it.

Assuming  the plasma to be collisionless and stating the accelerations in the plasma in terms of the electric field $\mathbf{E}$ and magnetic field $\mathbf{B}$, Equation~\ref{eqn:boltzmann} may be reduced to the Vlasov equation for a plasma interacting with electromagnetic fields.
\begin{equation}
\frac{\partial f}{\partial t}  +(\mathbf{v}\cdot\nabla_r)f + (\frac{q}{m}[\mathbf{E} + \mathbf{v}\times \mathbf{B}]\cdot\nabla_v)f = 0
\end{equation}
where $q$ is Coulomb charge, $m$ is particle mass, and $f(\mathbf{r}, \mathbf{v},t)$ is reduced to $f$ for simplicity of notation. This form of the Vlasov equation is used in the formulation of wave-particle interactions, and wave growth in plasmas, that ultimately lead to plasma emission in solar radio bursts, this will be described in further detail sections~\ref{sec:wave_particle}--\ref{sec:three_wave}.

In equations \ref{eqn:num_density} and \ref{eqn:bulk_flow}, the number density and bulk velocity of the plasma were obtained by taking appropriate weighted integrals of the distribution function. This is done to obtain information on the macroscopic properties of the plasma when the specific details of the particle distribution are not needed. A behaviour of the plasma on a macroscopic or fluid scale may be obtained by taking the appropriate integrals of the the Boltzmann equation in a procedure known as `taking the moments of the Boltzmann equation'. The moments of a function are given by 
\begin{equation}
\mu_n = \int x^n f(x) dx
\end{equation}
where the $n$ describe the moments e.g., $n=0$ is the {\it zeroth moment},  $n=1$ is the {\it first moment} etc. Taking the moments of the Boltzmann equation lead to a set of fluid conservation equations that describe the dynamics of a plasma on a continuum scale (no individual particle motion available). The moments are as follows
\begin{eqnarray*}
\int [\mathrm{Boltzmann~ eq.}]\times v^0 dv &\rightarrow& \mathrm{conservation~of~mass} \\
\int [\mathrm{Boltzmann~eq.}]\times v^1 dv &\rightarrow& \mathrm{conservation~of~momentum} \\
\int [\mathrm{Boltzmann~eq.}]\times \frac{v^2}{m} dv &\rightarrow& \mathrm{conservation~of~energy}
\end{eqnarray*}
In their most fundamental form, these conservation equations are used in what is known as a multi-fluid description of the plasma, where the dynamics and conservations of the various properties are treated seperately for each particle species e.g., electrons and protons will have different conservation equations and described separately as an \textquoteleft electron fluid' and \textquoteleft proton fluid'. However, combining these separate fluids into one \textquoteleft single fluid' framework constitutes a description of plasmas known as magnetohydrodynamics\footnote{The moments of the distribution function and the reduction of multi-fluid to single fluid conservation equations involve lengthy derivations that are not provided here but can be found in many texts \citep{goossens2003, inan2011}.}.

\subsection{Magnetohydrodynamics}\label{sec:12}
The conservation principles  derived from moments of the Boltzmann equation are firstly the mass conservation equation
\begin{equation}
\frac{D\rho}{Dt} = -\rho\nabla\cdot \mathbf{v}
\end{equation}
where $\rho$ is mass density, $\mathbf{v}$ is bulk flow velocity, $t$ is time, and $D$ represents a Lagrangian derivative ($D/Dt = \partial /\partial t + v\cdot \nabla$). This expression simply states that the rate of change of particles into or out of a volume is controlled by the fluid flow into and out of the volume. It has units of kg\,m$^{-3}$\,s$^{-1}$.

Secondly, the momentum conservation equation is
\begin{equation}
\rho\frac{D\mathbf{v}}{Dt}=-\nabla p + \mathbf{j}\times \mathbf{B} + \mathbf{f} +\nabla \cdot \mathbf{S}
\label{eqn:mhd_momentum}
\end{equation}
where $p$ is thermal pressure, $\mathbf{j}$ is the current density, $\mathbf{B}$ is the magnetic field, $\mathbf{f}$ are all body forces e.g., gravity, and $\mathbf{S}$ is a stress tensor. This equation shows that the change in momentum of a fluid may be due to a pressure gradient, the \textquoteleft j cross B' or Lorentz force, gravity, and any stresses in the plasma e.g., kinematic viscosity $\nu$. Note that with $\nabla \cdot \mathbf{S}=\rho\nu\nabla^2\mathbf{v} $, and setting gravity and the Lorentz term to zero gives the Navier-Stokes equation of motion of a viscous fluid. Note that all terms in this equation have the form of a body force (force per unit volume), and integration over volume would render this equation an explicit version of Newton's second law ($m\mathbf{a}=\mathbf{F}$). An important force for the eruption of CMEs is the Lorentz force. Using Equation~\ref{eqn:amperes_law}
\begin{equation}
\mathbf{j}\times \mathbf{B} = \frac{1}{\mu_0}(\nabla \times \mathbf{B})\times \mathbf{B} = \frac{1}{\mu_0}(\mathbf{B} \cdot \nabla)\mathbf{B} - \nabla \frac{B^2}{2\mu_0}
\end{equation}
The first term on the right is known as magnetic tension, and acts to restore a bent or curved field line to a straight line. The second term is known as the magnetic pressure and, like thermal pressure, any gradient in the field will produce a force. These two components of the Lorentz force play an important role in CME eruption.

The third conservation equation is for energy
\begin{equation}
\rho\frac{De}{Dt} + p\nabla\cdot \mathbf{v}=\nabla\cdot(\kappa\cdot\nabla T) +  \eta_e\mathbf{j}^2 + Q_{\nu} - Q_r
\end{equation}
where 
\begin{equation}
e = \frac{p}{(\gamma-1)\rho}
\end{equation}
is the internal energy per unit mass, $\gamma$ is the ratio of specific heats, $\kappa$ is the thermal conductivity, $T$ is the temperature, $\eta_e$ is electrical resistivity, $Q_{\nu}$ is heating by viscous dissipation, and $Q_r$ is a radiative term. This equation demonstrates that any changes in the fluid internal energy are due to divergence in the flow field, conduction, ohmic and vicious dissipation, and radiative losses.

The fluid momentum equations can be used to formulate Ohm's law for a plasma which describes the behaviour of current in terms of electric and magnetic fields and the fluid flow velocity
\begin{equation}
\mathbf{j} = \sigma(\mathbf{E} + \mathbf{v}\times \mathbf{B})
\end{equation}
This is a reduced version of the generalised Ohm's law and does not contains terms for the Hall effect, ambipolar diffusion, and electron intertia. The set of equations (2.11 - 2.15), combined with an equation of state 
\begin{equation}
p = nk_BT
\end{equation}
results in a fully closed set of variables and equations, which can be solved for any fluid or electromagnetic property. In order to further simplify these set of equations the electric field $\mathbf{E}$ and current $\mathbf{j}$ are often eliminated by using Faraday's law in combinations with Amp\'{e}re's and Ohm's law to produce the induction equation
\begin{equation}
\frac{\partial \mathbf{B}}{\partial t} = \nabla \times(\mathbf{v}\times \mathbf{B}) + \eta\nabla^2 \mathbf{B}
\end{equation}
which describes the evolution of the magnetic field in terms of the plasma flow and the magnetic diffusivity $\eta = \eta_e/\mu_0$.  The mass, momentum, internal energy equation, the induction equation, and the solinoidal constraint then define  a fully closed system of resistive MHD equations in terms of the variables ($\mathbf{B}, \mathbf{v}, p, \rho$)
\begin{eqnarray}
\frac{D\rho}{Dt} = -\rho\nabla\cdot \mathbf{v} \\
\rho\frac{D\mathbf{v}}{Dt}=-\nabla p + \frac{1}{\mu_0}(\nabla \times \mathbf{B})\times \mathbf{B} + \mathbf{f} \\
\frac{D\rho}{Dt} = -\gamma p\nabla\cdot \mathbf{v} +(\gamma -1)\frac{\eta}{\mu_0^2} (\nabla \times \mathbf{B})^2 \\
\frac{\partial \mathbf{B}}{\partial t} = \nabla \times(\mathbf{v}\times \mathbf{B}) + \eta\nabla^2 \mathbf{B} \\
\nabla\cdot \mathbf{B} = 0
\end{eqnarray}
It is also assumed here that the fluid is inviscid and not subject to any conductive or radiative energy losses or gains. Under this reduced set of equations, the ideal MHD equations are simply produced by setting $\eta = 0$ e.g., zero resistivity, which would eliminate the second terms on the right in Equations 2.21 and 2.11. Under such an assumption the ideal induction equation
\begin{equation}
\frac{\partial \mathbf{B}}{\partial t} = \nabla \times(\mathbf{v}\times \mathbf{B})
\end{equation}
may be used to show the \textquoteleft frozen flux' condition. This shows that when $\eta = 0$ the field is advected with the plasma or the field is \textquoteleft frozen-in' into the plasma and follows the flow velocity i.e., there is no `slipping' of the magnetic field with respect to the fluid flow.  On the other hand, a finite $\eta$ leads to a variety of important diffusive effects for the magnetic field. The most important of these diffusive processes is magnetic reconnection, which plays an extremely important role in solar flares and CMEs.



%-----------------------------------------------------------%
%		   Magnetic Reconnection	                 % 
%-----------------------------------------------------------%
\subsection{Magnetic Reconnection}\label{sec:13}

If the dissipation of concentrations of magnetic flux in the solar corona are due to diffusion processes only, a timescale for this diffusion process may be estimated from the magnetic induction equation. Assuming the diffusion term dominates over advection term we obtain
\begin{equation}
\frac{\partial B}{\partial t} = \eta\nabla^2\mathbf{B}
\label{egn:diff}
\end{equation}
As an order of magnitiude estimate for the derivatives we replace $\partial/\partial t \rightarrow 1/\tau_D$ and $\nabla^2 \rightarrow 1/L^2$. Equation~\ref{egn:diff} then gives
\begin{equation}
\tau_D = \frac{L^2}{\eta}
\label{eqn:diff_time}
\end{equation}
The typical length scales for a coronal active region are $10^7$\,m, and assuming a Spitzer magnetic diffusivity $\eta = 10^9T^{-3/2} = 1$\,m$^2$\,s$^{-1}$ \citep{spitzer1962}, equation~\ref{eqn:diff_time} gives a diffusive timescale of 3.1 million years.
Given that active regions evolve quasi-statically over weeks and evolve dynamically over timescales of seconds minutes during an eruptive process, there must be an evolution process other than a large scale diffusion in the solar atmosphere. Equation~\ref{eqn:diff_time} indicates that there may be two different ways in which the evolution of magnetic fields may occur over a much shorter times scale: either the typical length scale $L$ is much shorter, or magnetic diffusivity $\eta$ is much larger. The process by which short timescale field evolution takes place is magnetic reconnection.

Magnetic reconnection is the dynamical process that describes the short timescale evolution of magnetic fields in the corona (and other astrophysical and laboratory systems). It involves a change in magnetic topology that converts magnetic energy in kinetic and thermal energy of the bulk plasma and accelerates particles to high speeds. Magnetic reconnection typically occurs in a boundary between two oppositely directed magnetic field regions (Figure~\ref{fig:recconection}). At the very center of the boundary layer (called the diffusion region) the magnetic field must go to zero to account for a continuous change from positive to negative magnetic field. The total pressure balance in the boundary layer and on either side is \citep{asch2004}
\begin{equation}
B_1 + p_1 = p_{dr} = B_2 + p_2
\end{equation}
where $p_{1,2}$ and $B_{1,2}$ represent thermal and magnetic pressure on either side of the diffusion region, respectively, and $p_{dr}$ is the pressure inside the diffusion region (no magnetic pressure).


\begin{figure}[!t]
\begin{center}
\includegraphics[scale=0.7]{reconnection.pdf}
\caption[Sweet-Parker and Petscheck reconnection models]{The reconnection models of Swee-Parker (top) and Petscheck (bottom). The Sweet-Parker mechanism employs a long and thin ($\Delta>>\delta$) current sheet as the diffusion region. It produces reconnection rates that are much too slow to describe flare and CME energy dissipations rates. Petscheck proposed a similar model but with a diffusion region that is much smaller, with $\Delta\approx\delta$, resulting in reconnection rates that are consistent with flare and CME timescales. The dissipation of energy in this model is partly controlled by the presence of two slow-mode shocks which separate the sub-Alfv\'{e}nic inflow region and the super-Alfv\'{e}nic outflow region \citep{asch2004}.}
\label{fig:recconection}
\end{center}
\end{figure}
The existence of inflows into the diffusion region at right angles to the magnetic field creates an electric field in this layer that is perpendicular to the inflow plane. This creates a current in the diffusion region, hence this ayer is often called a `current sheet'. The diffusion of magnetic fields occurs within this current sheet. Sweet and Parker were the first devise an MHD mechanism by which magnetic reconnection occurs \cite{sweet1958, parker1963}. 
%Although no complete analytical theory exists for the process, the MHD conservation equations may be applied to derive the reconnection rate. The application of resistive MHD only applies within the thin current sheet, while ideal MHD applies everywhere else.
They showed that the diffusive time scale of the field (equation~\ref{eqn:diff_time}) could be made much faster by considering a diffusion region much smaller than the size scales of active regions. The model consists of a current sheet (diffusion region) that is much longer than it is wide i.e., in the top panel of Figure~\ref{fig:recconection} $\Delta>>\delta$, where $\Delta$ is current sheet half-length and $\delta$ is current sheet half-thickness. Conservation of mass implies that the rate at which mass enters the sheet must equal the rate that mass exits the sheet
\begin{equation}
\rho \Delta v_{in} = \rho \delta v_{out}
\label{eqn:in_out}
\end{equation}
where $v_{in}$ and $v_{out}$ are the inflow and outflow speeds respectively. Using equations~\ref{eqn:diff_time} and \ref{eqn:in_out} the inflow speed may be written as
\begin{equation}
v_{in}^2 = \frac{\eta v_{out}}{L}
\end{equation}
The reconnection rate depends on in the ratio of the inflow speed $v_{in}$ to the outflow speed $v_{out}$ in the form of the Mach number
\begin{equation}
M = \frac{v_{in}}{v_{out}} = \frac{1}{\sqrt{S}}
\end{equation}
where $S=v_AL/\eta$ is the Lundquist number (equivalent to the magnetic Reynold's number at the Alfv\'{e}n speed). Hence, the rate of reconnection depends on the length scale and magnetic diffusivity in the current sheet. Despite the fact that the Sweet-Parker mechanism provides a rate of magnetic energy dissipation that is faster than the global process, it is much too slow to explain the process of magnetic energy release in solar flares. For example for typical Lundquist numbers of $10^{12}$, the Sweet-Parker model produces a reconnection rate of $10^{-6}v_a$.

To over come the problem, Petscheck proposed a model with a much smaller diffusion region where $\Delta\approx\delta$ \citep{petschek1964}, see bottom panel of Figure~\ref{fig:recconection}. With this smaller diffusion region, the boundary between inflow and outflow regions are separated by slow mode magnetoacoustic shocks. Alongside the diffusion region, these shocks also help to dissipate some of the inflowing kinetic energy into thermal energy. Petscheck found the reconnection rate is given as
\begin{equation}
M \approx \frac{\pi}{8\,\mathrm{ln}(S)}
\end{equation}
producing a much faster rate of 0.1$v_A$, which is comparable to solar flares. Much work has been done on the generalization of this theory \citep{priest1986, sonnerup1970}, and there is observational evidence for the existence of reconnection in the corona \citep{su2013}
%Removed to reduce pages
%When Petscheck presented the model it was thought to be a complete description of 2D magnetic reconnection, however other solutions would soon be discovered, with the Petscheck model being just one of a family of possible 2D reconnection mechanisms. These mechanisms describe steady, uniform, 2D reconnection, the rate of which may be summarised as \citep{priest1986}
%\begin{equation}
%\bigg(\frac{M_e}{M_e}\bigg)^2 \approx \frac{4M_e(1-b)}{\pi}\bigg[ 0.834 -\mathrm{ln \, tan}\bigg( \frac {4S\sqrt{M_eM_i}} {\pi}  \bigg)^{-1}\bigg]
%\end{equation}
%This contains the Sweet-Parker, the Petscheck ($b=0$) and Sonnerup \citep{sonnerup1970}($b=1$) reconnection rates.

%While these solutions show that 2D reconnection may be explained well by MHD, describing 3D reconnection remains big theoretical and computational challenge, and remains at the forefront of reconnection theory research. Both 
%2D reconnection, an the more complicated 3D models, play a role in many theoretical models describing the eruption of coronal mass ejections and flares. Although some CME models have no need for reconnection to occur, there is a growing consensus that it is an integral part of the solar eruptive process.

%-----------------------------------------------------------%
%		       CME Theory			        % 
%-----------------------------------------------------------%

\section{Coronal Mass Ejections}\label{sec:2}

Magnetohydrodynamic models of eruptive coronal mass ejections use either ideal or resistive MHD (without and with reconnection, respectively) to bring about a loss of equilibrium of some complex magnetic structure in the corona. The magnetic structure usually takes the form of a flux rope, helical or twisted magnetic fields embedded in the coronal magnetic field. The main goal of every model is to induce a loss of equilibrium of the structure, but the mechanism by which this is done varies greatly amongst the models. {\it Storage models} assume a slow build up of magnetic stress in a non-potential field that may store free energy over long time scales before some loss of equilibrium occurs and the stored magnetic energy is rapidly converted to mechanical energy and the expulsion of a magnetic structure \citep{wolfson1998, forbes1995, antiochos1999}. {\it Dynamo models} involve a rapid generation of magnetic flux by either stressing of the field or flux-injection into the system. As the name suggests, these models usually consider the interplay between current and magnetic field in the system that may bring about a Lorentz force which provides an expulsion of the flux rope from the low corona \citep{chen1989, krall2001, schrijver2008, fan2005}. Finally there are the {\it thermal blast models}, which produce an expulsion of the CME into interplanetary space by an enhanced pressure gradient due to the rapid heating of a flare i.e., an explosive ejection of plasma from the corona.  This model is somewhat out-dated now since CMEs are no longer thought to be the result of flares, with some CMEs preceding flare onset and some occurring without any flare \citep{gosling1993}. The most prominent of the various models are discussed here.

%The main goal of every model is to induce a loss of equilibrium of the structure, but the mechanism by which this is done varies greatly amongst the models e.g., there is still some debate on whether the flux rope is formed as a result of eruption or it is a pre-existing structure. The need for reconnection is also a source of contention, with some models inducing eruption using only ideal MHD and other models employing resistive MHD \citep{chen2011}. The most prominent of these models are discussed here.

\subsection{Catastrophe Model}\label{sec:cat_model}
\begin{figure}[!t]
\begin{center}
\includegraphics[scale=0.4, trim=1cm 1cm 0cm 1.5cm]{images/catastrophe}
\caption[The catastrophe CME model]{The catastrophe model of \citep{forbes1995}. The model consists of a 2D pre-existing flux rope with foot points rooted in the photosphere. The fluxrope is driven toward instability by motions of the photospheric footpoints, in this case the distance between the footpoints $\lambda$ decreases slowly (timescales much longer than the Alfv\'{e}n crossing time $\tau=L/v_A$). As the foot points converge the fluxrope initially contracts indicated by a decreasing height in panel (a). Eventually this convergence brings the system to critical point where magnetic pressure outwards dominates inward magnetic tension. The system rises, reaches a new equilibrium position, and forms a current sheet. The evolution of the system after it reaches this new equilibrium largely depends on whether or not magnetic reconnection occurs in the sheet. the rate of reconnection may also bring about different evolutions in kinematics \citep{priest2000}.}
\label{fig:catastrophe}
\end{center}
\end{figure}

The catastrophe model assumes a flux-rope is formed in the corona prior to eruption and considers the balance between magnetic tension holding the flux rope in position, and magnetic pressure (from compression of field lines under the rope) that supply an outward directed force \citep{forbes1991, lin2000, priest2000}. A loss of equilibrium is brought about by photospheric motions, either convergence or shearing of the foot points, which are well-known precursors to eruptive activity in the corona \citep{rust1972}. The reduction of the distances between the foot points, $2\lambda$, decreases and this initially 
causes an increase in the magnetic tension which makes the rope contract and reduce its height Figure~\ref{fig:catastrophe}. However, continued contraction results in a magnetic compression that eventually dominates tension, resulting in a flux rope rise. As the rope rises it forms a current sheet behind it, and its evolution after this point depends on whether or not reconnection occurs in the current sheet. If no reconnection is present then the flux rope simply rises and finds a new equilibrium position at a greater height, in this case the net release of magnetic energy is less than 1\% of the energy stored in the pre-field configuration \citep{forbes1991}. If reconnection occurs, then the eruption proceeds uninhibited and up to 95\% of the stored magnetic energy is released \citep{forbes1995}.

\citet{forbes1995} provided expressions for the development of current in the flux rope with respect to height which was used to estimate the free magnetic energy in the system. By assuming a rapid reconnection rate, and that all of this free energy was converted to the rope's kinetic energy they were able to derive velocity-time kinematics, and under the constraint of the flux rope radius $a\rightarrow 0$ an analytical expression for the rope velocity may be derived as \citep{priest2000}
\begin{equation}
v\approx \sqrt{  \frac{8}{\pi}  }v_{A0}\bigg[\mathrm{ln}\bigg( \frac{h}{\lambda_0}\bigg) + \frac{\pi}{2}  - 2\mathrm{tan}^{-1} \bigg( \frac{h}{\lambda_0}\bigg)\bigg] + v_0
\end{equation}
where $h$ is the fluxrope height, $2\lambda_0$ is the foot point separation at $\lambda=h$, $v_0$ is an initial perturbation velocity (1\% of the Alfv\'{e}n speed), and $v_{A0}$ is the Alfv\'{e}n speed where $\lambda=h$. Magnetic power output in the early phase of eruption is given by
\begin{equation}
\frac{dW}{dt} \approx -\frac{2A_0^2}{\pi\mu}\bigg( \frac{h}{\lambda_0} -1\bigg)^2\frac{v}{\lambda_0}
\end{equation}
where $h\sim t + t^{5/2}$ and $v\sim t^{3/2}$ i.e., the initial power output grows with time. In the later phases of propagation the power output decays with time as
\begin{equation}
\frac{dW}{dt} \approx \frac{4A_0^2}{\pi \mu t}
\end{equation}
so the growth in power output occurs approximately 100 times quicker than the decay in power output.

A later study by \citet{priest2000} analysed how reconnection in the underlying current sheet may influence the eruption of the flux rope. The kinematics of the rope after equilibrium is lost depend on the rate of reconnection in the sheet, parameterised by the Alfv\'{e}n Mach number of the inflow into the reconnection site. If $M_A=0$ then the fluxrope does not escape but oscillates around an equilbrium height like a yo-yo. If $0<M_A<0.005$, escape is possible but the rope may show a number of oscillations in height before escape, this behaviour has never been directly observed so reconnection must occur at a rate $M_A>0.005$ to produce eruption.  For $0.005<M_A<0.041$ to rope escapes but undergoes a period of deceleration between 20 and 100 Alfv\'{e}n crossing times, while for $M_A > 0.041$ no deceleration occurs and the fluxrope escapes and approaches an asymptotic velocity.

The catastrophe model provides a successful way of evolving a flux system to the point of catastrophic loss of equilibrium and consequent eruption. However, a major limitation is that it is a 2D model and does not take into account that the ends of the flux rope will be anchored in the photosphere. This would produce a curvature in the rope that would increase its tension and hence change the dynamics, but it is unlikely that it would prevent eruption \citep{steele1989}




\subsection{Magnetic Breakout Model}\label{sec:21}

The magnetic breakout model was first proposed by \citep{antiochos1999} and involves a quadrupolar (or more complex) magnetic flux system. A core magnetic field is flanked by two side-lobe fields, which collectively lie underneath an over-arching field that stabilizes the whole system. The overarching field and core field are almost anti-parallel, creating a magnetic null point between the two (Figure~\ref{fig:breakout_model}). Non potentiality is injected into the core by twisting/shearing of the foot points or by flux emergence. This non-potentiality causes the core field to grow and encounter the overarching field, distorting the null point into a current sheet and eventually allowing reconnection to occur. The reconnection removes field lines from the overarching field and adds it to the side-lobe systems, allowing further growth of the core field. The growth of the core field in turn drives further breakout reconnection resulting in a positive feedback required for explosive expulsion of the core. Finally, as the core is accelerated a current sheet forms in its wake, eventually leading to a separation of the core flux from the solar surface that forms a plasmoid structure typical of a three part CME \citep{lynch2004}; an important aspect of this is that flux rope formation happens as a consequence of eruption i.e., it is not pre-existing. The magnetic breakout model was used to circumvent the Aly-Sturrock limit \citep{aly1991, sturrock1991} i.e., it allowed a flux system to erupt, without having to open the constraining field lines to infinity.
\begin{figure}[!t]
\begin{center}
\includegraphics[scale=0.45, trim=0cm 0cm 0cm 1cm]{images/lynch_breakout}
\caption[The breakout CME model]{The breakout model, consisting of a quadrupolar flux system in which the central flux (blue) is flanked by to side lobe flux systems (green), with the entire system kept in stability by the tension of the overlying red field. Shearing and/or twisting on the underlying flux causes it to grow slowly. Eventually a current sheet forms at the magnetic null above the central flux, causing reconnection. This reconnection transfers overlying field to the side-lobes, effectively creating a conduit for the central flux to escape as a CME \citep{lynch2008}.}
\label{fig:breakout_model}
\end{center}
\end{figure}

Kinematically, the CME/central field system should experience a slow rise (1\,km\,s$^{-1}$) for several hours due to shearing/twisting of the foot points. Once breakout reconnection has begun the CME experiences a much larger acceleration (100\,km\,s$^{-1}$). The reconnection in the current sheet in the wake of the CME is the source of energetic particles that ultimately lead to flaring (ribbons and soft x-ray loops). Therefore magnetic breakout predicts that the flaring process and SXR peak should only begin after CME acceleration (after breakout reconnection) has begun \citep{lynch2004}. However, the precedence in breakout reconnection over flaring reconnection may not always be the case, with the latter sometimes driving the former \citep{macneice2004}. The above studies have mainly been through 2.5\,D simulations but a 3\,D simulation of the breakout model was given in \citep{lynch2008}. This allowed a full estimate of the conversion of magnetic energy into kinetic energy. It is found that during the flare impulsive phase 17.8\% of the free magnetic energy ($4.6\times10^{31}$\,ergs) is converted into plasma kinetic energy ($8.1\times10^{30}$\,ergs). During the gradual phase the proportion of free magnetic energy converted to kinetic energy drops to 15.4\%
%Took out due to large page numbers
%There has been observational tests of the magnetic-breakout model, showing it to be a viable explanation of some flaring and CME events, the most notable of which is the Bastille Day event \citep{aulan2000}. The observational signatures of the model include the presence of a null point in the corona above a complex multipolar flux system (inferred from potential field source surface extrapolations), a radio source imaged to be above the erupting structure (implying a reconnection site), and radio bursts beginning at frequencies indicative of high altitude (again indicating energy release above the erupting structure, prior to eruption) \citep{manoh2003}. However, in some instances magnetic breakout is implied by observations of the above, but the kinematics are inconsistent with model predictions. For example the model predicts a long slow rise of the central flux system as the underlying field is increasingly sheared, after which there is a rapid acceleration once breakout reconnection is initiated. However, in the study of \citet{bong2006} the breakout reconnection occurred at the end of the CME acceleration phase, prompting a two-phase acceleration scenario.


\subsection{Toroidal Instability}\label{sec:ti_model}

The toroidal instability model incorporates a pre-existing flux rope structure that is built from a torus of magnetic flux, some of which is buried beneath the photosphere \citep{chen1989}. The flux system is can be broken down into a combination of toroidal magnetic field, toroidal current, poloidal magnetic field and poloidal current (Figure~\ref{fig:chen_model}). This flux rope system is embedded in a surrounding coronal magnetic field $\mathbf{B}_{corona}$. The stability of the system depends on the nature of the $\mathbf{J} \times \mathbf{B}$ force due to the interaction toroidal and poloidal components of both the field and current. The interaction of $\mathbf{J}$ and $\mathbf{B}$ internal to the flux rope is usually termed the Lorentz self-force or the \textquoteleft hoop' force. An instability may be induced via twisting of the fluxrope footpoints to increases the amount of poloidal flux (effectively increasing the helicity of the system). The instability arrises when the outward hoop force decreses more slowly within the ring radius than the opposing Lorentz force due to an external magnetic field. Once the instability is induced, the fluxrope begins a bulk motion as well as a growth in its semi-minor axis. Hence the motion of the system can be analysed by looking at the central axis or the minor axes (leading and trailing edges). The three axes display slightly different kinematics e.g., the leading edge has a faster velocity than the trailing edge (due to fluxrope expansion). This has proved a useful test of the model when comparing the observations of erupting fluxrope structures as seen in white-light coronagraphs. \citet{krall2001} analysed the leading and trailing edges of erupting flux rope, as well as the rope aspect ratio, and compared the observations to model expectations. Good agreement is found between the model kinematics and aspect ratio and the observed events.
\begin{figure}[!t]
\begin{center}
\includegraphics[scale=0.4, trim=0cm 1cm 0cm 1cm]{images/chen_model}
\caption[The toroidal instability CME model]{The flux rope model of \citet{chen1989}, used to to study the toroidal instability of a twisted flux system in the corona.}
\label{fig:chen_model}
\end{center}
\end{figure}
The equation of motion of the entire system is given by
\begin{equation}
M\frac{d^2Z}{dt^2} = \frac{I_t}{c^2R}\times\bigg[ \mathrm{ln}\bigg(\frac{8R}{a}\bigg) -1+ \frac{\xi_i}{2} + \frac{\beta_p}{2} -\frac{B^2_t}{B^2_{pa}}  -\frac{2RB_{\perp c}}{aB_{pa}} \bigg] - F_g - F_{drag}
\end{equation}
where $I_t$ is the toroidal current, $R$ is the flux rope major radius, $a$ is the rope minor radius, $\xi_i$ is internal inductance of the flux system, $B_t$ is the toroidal field, $B_{pa}$ is the poloidal field at $a$, $B_{\perp c}$ is the perpendicular component of the ambient coronal field, $F_g$ is the force due to gravity, $F_{drag}$ is the drag force, $M$ is the mass per unit length of the rope, and $Z$ is the rope axis height above the photosphere. The equation of motion shows that an increase in the toroidal current (or poloidal flux) contributes positively to the acceleration. The terms in the square brackets are each unitless and take into account the rope geometry, self-inductance and interplay between poloidal and toroidal flux. The first three terms in the square brackets are what give rise to the hoop-force. If the rope is mass loaded with a prominence, this can contribute to the rope's stability via the gravity term. The drag term only becomes an important contributor to rope dynamics later in the propagation, when the solar wind speed begins to increase i.e., at around 10$R_{\odot}$ \citep{sheeley1997}. The eruption is driven by flux-injection, which typically lasts for 4-8 hours, during which time the unstable system loses its equilibrium and begins to rise \citet{krall2001}. This model is perhaps the only one to calculate the magnitudes of the forces explicitly (Figure~\ref{fig:chen_forces})
\begin{figure}[!t]
\begin{center}
\includegraphics[scale=0.4, trim=0cm 1cm 0cm 2cm]{chen_forces}
\caption[The toroidal instability forces]{Forces acting on a flux rope from the Toroidal instability model. The y axis units are in 10$^21$ dyn. The dashed curve is the new force from all contributions \citep{chen1989}.}
\label{fig:chen_forces}
\end{center}
\end{figure}

It is significant the fluxrope is already established in the corona before eruption begins i.e., the rope formation is not addressed in the model and it is not a consequence of eruption. Hence magnetic reconnection is not a necessary aspect of the model and the eruption may proceed without employing resistive MHD. The model has been tested against observations and found to provide consistent result with the acceleration and jerk profiles of destabilized filaments during eruption \citep{schrijver2008}

%\subsection{Drag Models}\label{sec:23}
\section{Coronal Shocks and Plasma Emission}\label{sec:3}

The following is an overview of general coronal shock theory and the mechanisms by which a shock will produce plasma emission and radio busrsts. This includes shock drift acceleration, beam-wave interaction and the three-wave interaction process. 

\subsection{Alfv\'{e}n Speed in the Corona}
The speed at which perturbations travel in a magnetized plasma is the Alfv\'{e}n speed, given by
\begin{equation}
v_A = \frac{B_0}{\sqrt{\mu_0 \rho}}
\end{equation}
where $B_{0}$ is the unperturbed (equilibrium) magnetic field, $\mu_{0}$ is the magnetic permeability, and $\rho_{0}$ is the unperturbed mass density of the medium. This is a highly anisotropic wave driven by the restoring force of magnetic tension, with the inertia provided by the plasma mass density. Perturbations in the magnetic field, $B_{1}$, are transverse to the direction of $B_0$ and the group velocity always has a wave $\hat{k}$ vector parallel to the magnetic field direction. The quiet solar corona in the range of $1-3$\,$R_{\odot}$ has typical magnetic field strengths on the order of 1$-$100\,G\,=\,10$^{-2}-$10$^{-3}$\,T, and typical electron number densities of 10$^{6} -$10$^{9}$\,cm$^{-3}$\,=\,10$^{12}-$10$^{15}$\,m$^{-3}$, so for $B_0$=5 G and $n_e$=10$^{8}$\,cm$^{-3}$ the Alfv\'{e}n speed in the corona is $\sim$1000\,km$\cdot$s$^{-1}$. The variation in magnetic field and density in the corona, especially nearby an active region, $v_{A}$ may be on the order of 10$^{2}-$10$^{3}$\,km$\cdot$s$^{-1}$. If displacement components of the wave are perpendicular as well as parallel to the equilibrium magnetic field and non-zero perturbations in plasma thermal pressure and density occur, the result is a wave propagation known as a magnetoacoustic wave, such as the fast and slow mode MHD waves. However, given the corona is a low$-\beta$ plasma, magnetic perturbations are faster than thermal pressure ones, so the Alfv\'{e}n speed is a good estimate of plasma perturbation speeds in the corona.
\begin{figure}
\begin{center}
\includegraphics[scale=0.25, trim=0cm 2cm 0cm 2cm]{alfven_speed.png}
\caption[Model of the Aflv\'{e}n speed as a function of height in the corona]{Alfv\'{e}n speed of the corona as a function of heliocentric distance. The dotted line is the quite sun, with only the Sun's global dipole field. The solid line is the Alfv\'{e}n speed calculated from a combination of te global dipole field and a smaller active region dipole field oriented parallel to the global dipole. The dashed line is the the speed when the active region dipole is anti-parallel to the global field. The two profiles from the active region show a distinct minimum in the coronal Alfv\'{e}n speed at $\sim1.5\,R_{\odot}$}
\label{fig:alfven_speed}
\end{center}
\end{figure}
\citet{mann2003} produced a 1D model of the variation of Alv\'{e}n speeds in the quiet and active region corona as a function of height (Figure~\ref{fig:alfven_speed}), this was later extended to a 2D model \citep{warmuth2005}. It shows that CMEs are well capable of producing plasma hocks in the corona, since they may travel far in excess of the Alfv\'{e}n speed. The theory of plasma shocks and the resulting effects of particle acceleration and plasma emission are discussed in the following sections.


\subsection{MHD Shocks}\label{sec:mhd_shocks}

For acoustic shock waves there are a number of conservation equations that quantify the strength of a shock by relating the upstream gas pressure, density, flow speed, and temperature to their downstream counterparts. Such conservation equations are known as the jump conditions and the shock is considered a surface at which the fluid properties change discontinuously. A shock naturally converts kinetic energy to thermal energy by slowing the flow down and heating it i.e. a dissipation of kinetic energy and increase of entropy. This occurs within the extremely narrow shock surface itself by particle collisions and viscous interactions. However, the processes in the shock surface itself are inconsequential and the relationship between upstream and downstream macroscopic parameters provide a sufficient description of the plasma flow. 

When the gas is ionized and a magnetic field is present, the jump conditions are very much the same and are modified only by the presence of the magnetic field. In the general case of oblique shock waves where the magnetic field direction has some arbitrary angle with respect to shock normal we may derive a set of conservation equations for the frame of the shock wave. 
\begin{figure}[h!]
\begin{center}
\includegraphics[scale=0.5, angle=0]{images/shock_pic}
\caption[MHD shock framework]{Orientation of magnetic field and velocity field with respect to shock plane, in the rest frame of the shock. Shock normal in this case would be along the $-x$ direction i.e., into upstream region 1. \citep{fitz}}
\label{fig:shock_pic}
\end{center}
\end{figure}

The flow velocity $v$ and magnetic field $B$ are considered to be in the xy-plane. The appropriate conservation  equations are
\begin{subequations}
\begin{align}
&[\rho v_{x}]=0 \\
&[\rho v_{x}^2+p+\dfrac{B_{y}^2}{2\mu}]=0 \\
&[\rho v_{x}v_{y} - \dfrac{B_{x}B_{y}}{\mu}]=0 \\
&[\dfrac{1}{2}v^2 + \dfrac{\gamma p}{(\gamma-1) \rho}+\dfrac{ B_{y}(v_{x}B_{y} - v_{y}B_{x})}{\mu \rho v_{x}} ]=0 \\
&[B_{x}]=0 \\
&[v_{x}B_{y} - v_{y}B_{x}]=0 
\end{align}
\end{subequations}
These are the general MHD shock jump conditions where $v$ is the fluid velocity and $B$ is the magnetic field (with their corresponding components $\textquoteleft$x' or $\textquoteleft$y'), $\rho$ is the mass density, $p$ is the thermal pressure, and $\gamma$ is the ratio of specific heats (or the polytropic index). The meaning of the square brackets is $[F]\equiv F_{1}-F_{2}$, for any quantity F, and the $1$ or $2$ subscripts represent upstream or downstream values of each quantity F, respectively. These set of jump conditions differ only from the purely acoustic ones due the presence of the magnetic field. For example, taking (2.37a), (2.37b), and (2.37d) and setting $B_{x}=B_{y}=0$ we obtain the jump conditions for a neutral gas. Each conservation equation has a specific meaning; 
\begin{itemize}
\item (2.37a) is a mass conservation equation whereby the mass flux entering the shock must equal the mass flux leaving. It has units of kg\,m$^{-2}$\,s$^{-1}$.
\item (2.37b) indicates that if mass flux $\rho_{1} v_{x,1}$ enters the shock with momentum $(\rho_{1} v_{x,1})v_{x,1}$ it leaves the shock with momentum $(\rho_{1} v_{x,1})v_{x,2}$, the difference being equal to the the changing force per unit area across the shock. In this case both thermal and magnetic pressures contribute to change in momentum flux. (4c) implies the same process but relates the $x$ and $y$ components of the $v$ and $B$ vector fields. Both equations have units of momentum flux $\equiv$ (kg\,m$^{-2}$\,s$^{-1}$)(m\,s$^{-1}$) = (kg\,m\,s$^{-2}$m$^{-2}$) = N\,m$^{-2}\equiv$ pressure.
\item (2.37c) is an energy conservation term, accounting for the rate at which gas and magnetic pressure do work per unit area at the shock and equates this to the growth (or loss) in internal energy and kinetic energy across the shock. All components of magnetic field pressure are taken into in the last term on the left of the equation. All quantities are in units of J$\cdot$kg$^{-1}$.
\item (2.37d) simply states that the x component of the magnetic field i.e., the component of the field that is (anti-)parallel to the shock normal $\hat{n}$ is unaffected by the shock transition. 
\item (2.37f) relates the orientations of the upstream and downstream magnetic field to the flow speed tangential and perpendicular to the shock normal. Magnetic field orientation and hence the distribution of flow speed amongst the velocity components largely depends on the whether the shock is slow-mode, intermediate, or fast mode. The equation has units of T$\cdot$m$\cdot$s$^{-1}$ = V$\cdot$m$^{-1}\equiv$ electric field.
\end{itemize}

%2.37(a-f) are the general case of the jump conditions across an MHD shock, usually known as the MHD Rankine-Hugoniot (RH) equations. Their generality make the solution of the six unknowns from the six equations quite complicated. However the extreme cases of parallel and perpendicular shocks provide very useful and simplified expressions. It can be shown that parallel shocks i.e., $\hat{v}\parallel\hat{B}\parallel\hat{n}$ reduces to the jump conditions of a hydrodynamic shock in a neutral gas (here parallel and anti-parallel are used synonymously). The more interesting case when considering radiating shockwaves in the low solar corona is the perpendicular (or quasi-perpendicular) MHD shock, in this case the magnetic field is perpendicular (or at a high angle) to shock normal. As will be shown it is this special case of quasi-perpendicular shocks that lead to efficient shock particle acceleration, a necessary precursor to the generation of radio emission at a coronal shockwave. 
2.37(a-f) are the general case of the jump conditions across an MHD shock, usually known as the MHD Rankine-Hugoniot (RH) equations. To important classes of shock may defined: quasi-perpendicular where the angle $\theta$ between shock normal $\hat{n}$ and $\hat{B}$ is $>45^{\circ}$, and quasi-parallel where $\theta<45^{\circ}$. These two classes of shock behave quite differently, and we will concentrate on quasi-perpendicular shocks here, which are important for solar radio bursts.

In the simple case of fully perpendicular shocks there is no need for the decomposition of the magnetic and velocity vector fields, meaning the $x$ and $y$ subscripts on 2.37(a) and 2.37(b) can be dropped. 2.37(c) is an obsolete jump condition, likewise for 2.37(e) since no $B_x$ field exists. We can also rid the $B_{x}B_{y}$ terms from the last quotient in the energy conservation 2.37(d) --replacing it simply with $B^{2}/2\mu\rho$. 2.37(f) reduces to a simpler form of $[Bv]=0$. Such a reduction in the generalized jump conditions allows us to express the upstream and downstream plasma properties in terms of the shock compression ratio $\chi=\dfrac{\rho_{2}}{\rho_{1}}$ as well as the upstream sonic Mach number $M_{1}=\dfrac{v_{1}}{c_{1}}$ \citep{priest2000} e.g.,
\begin{subequations}
\begin{align}
&\dfrac{v_{2}}{v_{1}}=\frac{1}{\chi} \\
&\dfrac{B_{2}}{B_{1}}=\chi \\
&\dfrac{p_{2}}{p{1}}=\gamma M_{1}^2\bigg(1-\frac{1}{\chi}\bigg) - \frac{1-\chi^2}{\beta_{1}}
\end{align}
\end{subequations}
where $\beta_{1}=2\mu p/B_{1}^2$ is the upstream plasma beta parameter. The exact value of the compression ratio may be obtained by using 2.37(b) to eliminate $p$ from the energy flux equation 2.37(d) and incorporating 2.37(a,c,e,f) (and a lot of algebra) into a quadratic for $\chi$
\begin{equation}
2(2-\gamma)\chi^{2}+[2\beta_{1}+(\gamma-1)\beta_{1}M_{1}^2+2]\gamma\chi - \gamma(\gamma+1)\beta_{1}M_{1}^2=0
\end{equation}
%where $M_{1}$ is the upstream sonic Mach number, $\beta_{1}$ is the upstream beta parameter and, and $\gamma$ is the ploytropic index. 
Equation (6) has one positive real root such that 
\begin{equation}
1< \chi < \frac{\gamma+1}{\gamma-1}
\label{eqn:compression}
\end{equation}
Using a ploytropic index of $\gamma=5/3$ (monatomic) means the shock compression can be no more than a factor of 4. Another extremely important fact arising from this is magnetic compression can also be no greater than 4 i.e., from equation 2.37(b) $B_{2}/B_{1}=\chi<4$. This has consequences for the shock drift acceleration mechanism and provides an upper limit to the particle energy gain.

Despite the fact that the Rankine-Hugoniot equations for neutral gas shocks and plasma shocks are essentially the same, on a microscopic scale they are fundamentally different. For most astrophysical circumstances plasmas shocks are collisionless. This means the mean free path for particle collisions is far larger than the shock thickness and it is not collisions between particles that dissipate flow energy and increase entropy. It is believed that the interactions of particles and the turbulent electromagnetic field of the shock play the role of collisions and produce flow energy dissipation. However, if a plasma shock becomes `supercritical' an extra dissipation mechanism is required, achieved by reflection of ions back to the upstream flow. Supercriticality is achieved when the downstream flow surpasses sonic Mach 1.

\subsection{Shock Particle Acceleration}\label{sec:30}

The framework of shock particle acceleration for solar type II radio bursts is called the shock drift acceleration (SDA) mechanism \citep{holman1983}. The mechanism involves a gyrating particle encountering the magnetic gradient caused by the shock, resulting in a guide center drift and an energy gain due to the presence of a convective electric field at the shock. 

There are two important frames of reference through which SDA is studied. In the rest frame of the shock, known as the \textquoteleft normal incidence frame' (NIF), the upstream plasma has velocity $\mathbf{u}_1$ along the normal $\hat{n}$ to the shock front. The upstream magnetic field $\mathbf{B}_1$ creates an angle $\theta_{Bn}$ with the shock normal $\hat{n}$, and the downstream counterparts have values $\mathbf{u}_2$ and $\mathbf{B}_2$, Figure~\ref{fig:nif_dht}(left). 

\begin{figure}[!t] 
\begin{center}
\includegraphics[scale=0.45]{nif_dht}
\caption[Normal incidence and de Hoffman-Teller reference frames]{(Left) Normal incidence frame (NIF) where the shock is at rest and the upstream plamsa approaches the shock head-on at velocity $\mathbf{u}_1$. The magnetic field makes some arbitrary angle $\theta_{Bn}$ with the shock normal. (Right) Transformation to a de Hoffman-Teller frame ensures that the plasma velocity and magnetic field are in the same direction on both sides of the shock \citep{ball2001}.}
\label{fig:nif_dht}
\end{center}
\end{figure}

Due to the motion of the plasma across the magnetic field, there is a convective electric field given by $\mathbf{E} = \mathbf{v}\times \mathbf{B}$, which causes a drift of the particles with speed $\mathbf{v}_E = \mathbf{E}\times \mathbf{B}/B^2$. The kinematics of the particle in the shock drift acceleration mechanism are best treated in a frame where there the convective electric field vanishes such that $\mathbf{E} = |\mathbf{v}\times \mathbf{B}|=v_x B_y - v_yB_x = 0$. The frame where this criterion is fulfilled is known as the de Hoffmann-Teller frame (dHTf) \citep{dehoffmann1950} and has a frame velocity with respect to the NIF frame given by. 
\begin{equation}
v_{HT} = v_y = u_1\mathrm{tan}\theta_{Bn}
\end{equation}
This frame guarantees the plasma motion is in the same direction as the magnetic field on both sides of the shock. Figure~\ref{fig:nif_dht} shows the difference between the NIF and dHT frames. Given the absence of any electric fields in this frame, the particle motions may be treated as having a conserved magnetic moment \citep{ball2001}
\begin{equation}
\mu = \frac{mv^2_{1\perp}}{B_1} = \frac{mv^2_{2\perp}}{B_2} = \mathrm{const}
\label{eqn:ad_in}
\end{equation}
where the subscripts [1,2] represent pre an post-encounter shock values respectively. In the dHTf, the conservation of the magnetic moment may be used to derive the particle kinematics given either reflection or transmission though the shock. Rearrangement of equation \label{eqn:ad_in} shows that particles with a pitch angle fulfilling the relationship
\begin{equation}
\alpha > \alpha_c~~~~\mathrm{where}~~~~\mathrm{sin}^2\alpha_c = \frac{B_1}{B_2}
\end{equation}
will be reflected at the shock. This pitch angle $\alpha_c$ defines a \textquoteleft loss-cone' in velocity space $f(v_{\perp}, v_{||})$, whereby any particle within the cone (large $v_{||}$) will be lost downstream, while particles outside the loss cone will be reflected at the shock. This is known as a \textquoteleft magnetic mirroring', a process that shocks are known to exhibit \citep{feldman1983}. Inside the dHT reference frame, the particles energy (whether reflected or transmitted) is completely conserved and there is no energy gain. Conceptually, the best way to see where the acceleration takes place is in the normal indcidence frame (NIF).
\begin{figure}[!t] 
\begin{center}
\includegraphics[scale=0.3]{sda_original}
\caption[Shock drift acceleration]{Particle drift paths due during both a transmitted (a) and reflected (b) motion. The increased magnetic field in the downstream region caused a drift along the surface. This drift occurs in the presence of a convective electric field (which will have a component parallel to the drift), allowing the field to do work on the particle and hence increase its energy \citep{ball2001}.}
\label{fig:sda}
\end{center}
\end{figure}
In the NIF, the particles gyrate about the upstream magnetic field $\mathbf{B}_1$, while there is a convective electric field causes a small drift of magnitude $\mathbf{v}_E = \mathbf{E}\times \mathbf{B}/B^2$ towards the shock. If the particle speed is much greater than the shock speed $v>>u$, the particle undergoes many Larmor gyrations while in contact with the shock. The difference between the Larmor radius of the orbit ahead of and behind the shock front (due to the increased downstream magnetic field) will cause the particle to drift parallel to the shock surface \citep{ball2001, toptychin1980}, Figure~\ref{fig:sda}. This is equivalent to a \textquoteleft grad-B' drift in an inhomogeneous magnetic field. This drift allows a charged particle to move parallel to the convective electric field, allowing the E-field to do positive work on the particle and produce an energy increase. Overall, a grad-B drift at the shock surface gives the particle a component of velocity that may interact with the convective electric field, hence the process is known as \textquoteleft drift-acceleration'.


While the energy gain is most apparent in the NIF frame, particle reflection and transmission is best handled mathematically in the dHT frame. Hence firstly, the magnetic mirroring process is described in dHT, while the post-encounter speed and energy gain are then obtained by converting back to the rest frame of the upstream plasma (NIF). \citet{ball2001} has shown that the particle energy gain upon reflection from the shock is given by 

\begin{equation}
\frac{E_r}{E_i} = \frac{1+\sqrt{1-B_1/B_2}}{1-\sqrt{1-B_1/B_2}}
%v^r_{||} = 2u_1\mathrm{sec}\theta_1 - v^i_{||}
\end{equation}
The energy gain is limited to the magnetic field strength jump across the shock, and since this field strength is limited by equation~\ref{eqn:compression}, the energy gain is limited to a factor of 13.93 \citep{ball2001}. A similar treatment may also give the reflected velocity in terms of the incident velocity \citep{holman1983}
\begin{equation}
v^r_{||} = 2u_1\mathrm{sec}\theta_1 - v^i_{||}
\end{equation}
shock drift acceleration has been used to explain the presence of $1-100$\,keV electron at Earth's magnetospheric bow shock \citep{wu1984}, in the context of radio bursts have used it to explain the acceleration of electrons during type II and herringbone bursts \citep{holman1983, mann2005, schmidt2012b}. 

Shock drift acceleration has the ability to produce high speed electron beams at the shock front. The instability of the plasma due to these electrons results in the growth of Langmuir waves. The interaction of beams and waves is the next step in describing plasma emission.
%More sophisticated models involve multiple reflections, which are important in some theoretical treatments of herringbone emission; we will return to this point in Chapter 5.



\subsection{Wave-Particle Interaction}\label{sec:wave_particle}

%Distributions of particles in a plasma can give rise to resonances in various wave modes. In general we may relate a particle distribution function in velocity space $f(v_{\|})$ with a distribution function of wave modes in wave-number or $\textquoteleft\mathbf{k}$-space' $N(\mathbf{k})$
%\begin{equation}
%\frac{\partial N(\mathbf{k})}{\partial t}+v_{g}(\mathbf{k})\frac{\partial N(\mathbf{k})}{\partial r} = \Gamma(\mathbf{k},f(v_{\|}))N(\mathbf{k})
%-\Gamma_{coll}(\mathbf{k}) N(\mathbf{k})
%\end{equation}
%where the lefthand side takes the form of a Lagrangian derivative with group speed $v_g$, $\Gamma$ represents a wave growth term given the distribution function $f(v_{\|})$, and $\Gamma_{coll}$ is a term taking into account collisional damping of the wave \citep{aschbook}.
The growth of collective oscillations in a plasma in response to the presence of some distribution of particles $f(\mathbf{r}, \mathbf{v}, t)$ in phase space is treated most simply with `quasi-linear theory' \citep{vedenov1963}. This is essentially a perturbation theory using the Vlasov and Maxwell's equations and successfully described the growth of plasma oscillations in the presence of electron beams. In order to see how a particle distribution function effects wave growth we start with an equilibrium distribution $f_0$ and impose a perturbation $f_1(\mathbf{r}, \mathbf{v}, t)$, so that the total distribution function is $f(\mathbf{r}, \mathbf{v}, t) = f_0 +f_1(\mathbf{r}, \mathbf{v}, t)$, with $f_1<<f_0$.  The perturbation quantities will take the form $e^{i(\omega t + \mathbf{k}r)}$ i.e., perturbation quantities that are periodic in space and time with wave number $\mathbf{k}$ and angular frequency $\omega$.

To see how $f(\mathbf{r}, \mathbf{v}, t)$ evolves in time we insert it into the Vlasov equation and linearize, ignoring any terms higher than second order
\begin{equation}
\frac{\partial f_1}{\partial t} + (\mathbf{v}\cdot \nabla)f_1 +\frac{q_e}{m_e}(\mathbf{E} + \mathbf{v}\times \mathbf{B})\cdot\nabla_vf_0=0
\end{equation}
%Assuming an isotropic $f_0$ it can be shown that $(\mathbf{v}\times \mathbf{B})\cdot\nabla_vf_0=0$, and 
Replacing the time and space derivatives with their oscillatory operators ($\partial/\partial t \rightarrow -i\omega$, $\nabla \rightarrow i\mathbf{k}$), the perturbed Vlasov equation may be rearranged to give
\begin{equation}
f_1=\frac{q_e}{m_e}\frac{j}{\omega-\mathbf{k\cdot v}}\mathbf{E}\cdot\nabla_vf_0
\label{eqn:f_pert}
\end{equation}
This equation relates the perturbed quantity $f_1$ to the unperturbed distribution function and the electric field. The most important aspect of this equation is the $\omega-\mathbf{k\cdot v}$ term in the denominator, implying the possibility of resonance between waves of phase speed $\omega/\mathbf{k}$ and particles of speed $v$. Now, integrating both sides of \ref{eqn:f_pert} over all velocity space and again using Maxwell-1 results in
%using (19) the general form of the dispersion relation of the oscillations $(\omega,\mathbf{k})$ in perturbations $f_1$ and $E$ is derived
\begin{equation}
1+\frac{4\pi q_e}{m_e}\frac{1}{k^2}\mathbf{k}\cdot\int_v\frac{\nabla_v f_0}{\omega-\mathbf{k\cdot v}}d^3v=0
\label{eqn:dispersion}
\end{equation}
Any electrostatic normal mode oscillations satisfy this relationship and integrating over all velocity space will result in a dispersion relation for the oscillations of the plasma property $f_1$. The equation effectively gives the oscillatory response $\omega = g(\mathbf{k})$ of the perturbation $f_1$ of the plasma, given a particular velocity distribution function $f_0$. 

%For example, a cold plasma distribution function where $f_0 = N_0\delta(\mathbf{v})$, where $\delta$ is a Dirac-delta, in equation~\ref{eqn:dispersion} will result in a dispersion relation of
%\begin{equation}
%1 - \frac{N_0q_e^2}{\epsilon_0 m_e \omega^2} = 0 
%\end{equation}
%The perturbed Vlasov equation with a cold plasma model (no thermal motions) produces the expression for a plasma oscillation. 

The integral over velocity space in this case can be performed because the delta function guarantees that there is no velocity with the property of $v = \frac{\omega}{\mathbf{k}}$. If this were to be fulfilled, there is the possibility of resonances equation~\ref{eqn:dispersion} i.e., there is a singularity in the integral, requiring a non-trivial as $v \rightarrow \frac{\omega}{k}$. The solution requires integration is performed in complex space using a method called contour integration \citep{melrose1989}. The use of contour integration means there will be complex solutions of the dispersion relation of the form $\omega = \omega_r + i\gamma$.
%The real part of the solution applies as normal where the integral is well behaved, far from the singularity. However, integration over the singularity necessitates a complex solution (from the contour integration), hence the $\omega$ consists of both real and imaginary parts.
The complex $i\gamma$ means that a time dependency of the periodic solutions to the perturbations $f_1$ will be 
\begin{equation}
e^{j\omega t}=e^{(i[\omega_r + i\gamma])}=e^{i\omega_rt}e^{-\gamma t}
\label{eqn:growth}
\end{equation}
This solution is a damped wave with damping factor $\gamma$, meaning the solution to equation~\ref{eqn:dispersion} in the region $v \sim \omega/k$ provides a wave decay term. These are the essential elements of Landau damping e.g., if the phase speed of the waves in a plasma match the speed of electrons in the distribution function then those waves will experience a damping.

Quasi-linear theory uses Equation~\ref{eqn:f_pert} and Maxwell's first equation to show that the time evolution of the wave energy density in the electrostatic waves is
\begin{eqnarray}
\frac{\partial W(k,t)}{\partial t} &=& -\gamma W(k,t)~,~~~\gamma = -\frac{\pi}{n}\frac{\omega^3}{k^2} \frac{\partial f(v,t)}{\partial v}
\label{eqn:Wk}
\end{eqnarray}
where $W(k,t)$ is the spectral energy density of the electrostatic waves and $\gamma$ is the wave growth term \citep{vedenov1963}.
Simultaneously, the distribution function is evolves according to 
\begin{eqnarray}
\frac{\partial f(v,t)}{\partial t} &=& \frac{\partial }{\partial v} \bigg(D\frac{\partial f(v,t)}{\partial v}\bigg)~,~~~D = \frac{4\pi e^2}{m_e^2}\frac{W(k,t)}{v}
\label{eqn:Diff}
\end{eqnarray}
with $D$ being a diffusion coefficient. Equations~\ref{eqn:Wk}--\ref{eqn:Diff} are the closed set of quasi-linear theory equations describing the interaction of electrostatic waves with energy spectrum $W$ and particle distribution function $f$ \citep{vedenov1963, kontar2001}. The wave growth term $\gamma$ has an important dependence on the velocity space gradient. From Equations~\ref{eqn:Wk}, $\partial f/\partial v >0$ will result in $\gamma <0$ and a wave growth term that increases exponentially; this can also be seen Equation~\ref{eqn:growth}. Effectively, a group of electrons is promoted to a velocity range that closely matches the phase speed of some type of waves in a medium $v \sim \omega/k$. In such a speed range the integrand of Equation~\ref{eqn:dispersion} approaches the point at which there is a singularity in the solution. Such a point necessitates a complex solution to the integral, thus providing an imaginary part to the dispersion relation that gives an $e^{-\gamma t}$ term. If the promotion of electrons to a range $v \sim \omega/k$ is combined with a positive gradient of the distribution function in this velocity range, it results in $\gamma < 0$ and $e^{-\gamma t}>0$, leading to wave growth. This is why $v - \omega/k=0$ is known as the resonance condition \citep{melrose1989}. The exchange of energy between the wave spectral energy distribution and the particle distribution is then described by ~\ref{eqn:Wk}--\ref{eqn:Diff}.

For type II and III radio bursts, quasi-linear theory is usually applied to the interaction of electron beams and Langmuir waves \citep{kontar2001, reid2010, ratcliffe2012}. This group of electrons in the beam is called a bump-on-tail, since it is described by a Gaussian bump on the high velocity tail of a Maxwell-Boltzmann velocity distribution function (Figure~\ref{fig:bot}). 
\begin{figure}[!t] 
\begin{center}
\includegraphics[scale=0.2]{bump_on_tail}
\caption[Shock drift acceleration]{Maxwell-Boltzmann velocity distribution with a Guassian `bump-on-tail' representing the beam. The shaded region are those electrons that will produce a bump-on-tail instability due to the presence of a positive slope in the distribution function.}
\label{fig:bot}
\end{center}
\end{figure}
The growth of Langmuir waves in a resonant response to this beam is called the bump-on-tail instability. Once these Langmuir waves are generated they may undergo decay or coalescence with other waves to produce electromagnetic radiation. 


\subsection{Three-Wave Interaction and Plasma Emission}\label{sec:three_wave}

Once the Langmuir waves are produced from the bump-on-tail instability a number of wave interaction processes occur in order to bring about plasma emission. This involves the interaction of various wave modes in the plasma described by a mathematical formalism called the three-wave interaction \citep{robinson1993, robinson1994}. In this process three wave modes in a plasma M, P, and Q are described by their distribution functions in a wave-number space ($k$-space). the distribution functions are given by $N_M(k_M)$, $N_P(k_P)$, $N_Q(k_Q)$, where the $N$ describe the occupation number of wave quanta between $k$ and $k+dk$ in the wave-number space. Waves in P and Q mode may interact to such that wave quanta are removed from the P and Q k-space and added to the M k-space. This is essentially an emission of an energy packet from the P and Q -space to the M k-space. The rate of change of occupation numbers in the three k-spaces are given by
\begin{eqnarray}
\frac{dN_M(\mathbf{k}_M)}{dt} = -\int \frac{d^3\mathbf{k}_P}{(2\pi)^3}\int \frac{d^3\mathbf{k}_Q}{(2\pi)^3}g(\mathbf{k}_M, \mathbf{k}_P, \mathbf{k}_Q) \\
%
\frac{dN_P(\mathbf{k}_P)}{dt} = -\int \frac{d^3\mathbf{k}_M}{(2\pi)^3}\int \frac{d^3\mathbf{k}_Q}{(2\pi)^3}g(\mathbf{k}_M, \mathbf{k}_P, \mathbf{k}_Q) \\
%
\frac{dN_Q(\mathbf{k}_Q)}{dt} = -\int \frac{d^3\mathbf{k}_M}{(2\pi)^3}\int \frac{d^3\mathbf{k}_P}{(2\pi)^3}g(\mathbf{k}_M, \mathbf{k}_P, \mathbf{k}_Q)
\end{eqnarray}
where $g(\bf{k}_M, \bf{k}_P, \bf{k}_Q)$ is an expression that incorporates a transition probability for wave quanta into and out of energy states in the various k-spaces \citep{robinson1994}. The transition probability of waves amongst states M, P and Q is given by \citep{melrose1986}
\begin{equation}
u_{MPQ}(\mathbf{k}_M, \mathbf{k}_P, \mathbf{k}_Q)  \propto \delta(\omega_M - \omega_P - \omega_Q ) \delta^3(\mathbf{k}_M - \mathbf{k}_P - \mathbf{k}_Q )
\end{equation}
where the $\omega$ are the frequency of the corresponding wave and and $\delta$ are delta functions. This is analogous to transition probabilities given by the Einstein coefficients for transferring energy packets from and atomic state to a photon state (photon emission) i.e., whereas the Einstein coefficients are used in atom-wave (atom-photon) energy exchanges, $u_{MPQ}$ describes wave-wave energy exchanges. Given the presence of delta functions in the transition probability expression, we can see that an exchange of energy quanta amongst the wave modes can only occur when $\omega_M =  \omega_P +\omega_Q$ and $\mathbf{k}_M  =  \mathbf{k}_P + \mathbf{k}_Q$.
%Hence for an a conversion wave modes in a plasma such as $M \rightarrow P + Q$ (a decay of mode M into P and Q), or it's reverse process $P + Q \rightarrow M $ (a coupling of P and Q to produce M) is described by equations (2.1) to (2.7). 
%\begin{multline}
%g(\mathbf{k}_M, \mathbf{k}_P, \mathbf{k}_Q) = u_{MPQ}(\mathbf{k}_M, \mathbf{k}_P, \mathbf{k}_Q)   [N_M(\mathbf{k_M}) N_P(\mathbf{k_P})  - \\ 
%N_P(\mathbf{k_P}) N_Q(\mathbf{k_Q})   +N_Q(\mathbf{k_Q}) N_M(\mathbf{k_M})  ]
%\end{multline}
%where $u_{MPQ}(\bf{k}_M, \bf{k}_P, \bf{k}_Q) $ is the transition probability from states in P and Q  to M, for example \citep{melrose1986}.  The transition probability is given by

%where the $\omega$ are the frequency of the corresponding wave and and $\delta$ are delta functions. 
The production of plasma emission requires a three wave interaction amongst a Langmuir wave $L$, ion acoustic wave $S$, and electromagnetic wave $T$. Fundamental and harmonic emission occur via 
\begin{eqnarray}
&L& \rightarrow T + S \\
&L& + ~L^{'}\rightarrow T'
\end{eqnarray}
Fundamental results from the decay of a Langmuir wave to an EM wave and ion acoustic wave. Second harmonic is the product of two Langmuir waves propagating in the opposite directions. The decay and coalescence processes may only occur when
\begin{eqnarray}
\omega_T & = & \omega_L + \omega_S \\
\omega_{T^{'}} & = & \omega_L + \omega_{L^{'}}
\end{eqnarray}
where $L'$ represents a product Langmuir wave and $T'$ represents a second harmonic EM wave. The relevant dispersion relations are 
\begin{eqnarray}
\omega_L = \omega_p + \frac{3v_{th}^2}{2\omega_p}k_L^2 \\
\omega_T = (\omega_p^2 +k_T^2c^2)^{1/2} \\
\omega_S = k_s\sqrt{\frac{\gamma k_B T_e}{m_i}}
\end{eqnarray}
where
\begin{equation}
\omega_p = \bigg(\frac{n_e e^2}{m_e \epsilon_0}\bigg)^\frac{1}{2}
\label{eqn:plasma_frequency}
\end{equation}
The Langmuir wave has a frequency very close to the plasma frequency, while the ion acoustic frequency will be much smaller than this (due to the larger ion mass in the Equation 2.62). Therefore, Equations 2.58 and 2.59 result in $\omega_T \approx \omega_p$ and $\omega_T\approx 2\omega_p$ i.e., radiobursts often occur at the plasma frequency and its harmonic. Much of the theory is confirmed by in-situ observations have provided insight into the generation of ion acoustic waves from the electrostatic decay of Langmuir waves, followed by the production of radio emissions \citep{thejappa1998}. However, in the low corona the beam generation of Langmuir waves, ion acoustic waves, and ultimately radio emission is still a subject of research \citep{kontar2001, reid2010, ratcliffe2012}.
%!!!!!!!!!!!!!!!!!!!!!!!!!!!!!
%
%PROVE THE ABOVE MORE CLEARLY!!!!!!!!!!!!!!!!!!!!!!
%
%!!!!!!!!!!!!!!!!!!!!!!!!!!!!

In order to investigate the amount of energy emitted by the electromagnetic wave, elements of the three wave interaction theory need to be combined with what is known as stochastic growth theory \citep{robinson1993a}(SGT). 
%Stochastic growth theory was first developed in response to a number of criticisms of the theory of a beam-driven Langmuir wave hypothesis of type III radio bursts first proposed by \citet{ginzburg1958}. It was pointed out that if the beam was to remain in instability continuously over space and time in the absence of saturation mechanism, then it would quickly lose all of its energy to the Langmuir waves and consequently the beam electrons would stop propagating after a short distance \citep{sturrock1964}. This is clearly not the case, since type III electrons are observed to propagate over distances of 1\,A.U or more. 
%Secondly, if the instability continues uninhibited the Langmuir wave with grow to levels far beyond what is observed \citep{smith1979}. 
%It was concluded that the beam cannot remain in a state of instability for the duration of its propagation. One suggestion to over come this was quasi-linear relaxation \citep{grognard1975}, as described above. However, this may not be enough to guarantee a lone lifetime of the beam. 
%SGT overcomes this by describing an electron beam propagation in a state of marginal stability. The 
SGT described a beam propagating in an inhomogeneous medium whereby it encounters pockets of localised density enhancement where it is unstable only intermittently, this is to prevent the beam losing all of its energy too quickly \citep{sturrock1964, ginzburg1958}. As the beam propagates through this localised clump, it is unstable to the growth of Langmuir wave, giving up some of its energy to the wave, diminishing the beam slightly. Once it exits the clump it becomes stable again, allowing the beam to reform until it reaches another localised clump to give up some energy again. The Langmuir waves would then experience a growth at intermittent regions in space and time (which is actually observed \citep{lin1986}), while the beam would be unstable for only small fractions of its lifetime. Overall, the beam energy loss is on average low enough to allow its continued propagation, but instantaneous and finite enough to allow the growth of waves. 
%In this mechanism the Langmuir waves experience a a growth in energy that is a random (stochastic) walk through energy phase space \citep{robinson1992}.  The Langmuir wave send up with an energy distribution given by $P(E)\sim E^{-1}$ \citep{robinson1993a}.  
\begin{figure}[t!]
\begin{center}
\includegraphics[scale=1.1, trim=0cm 0cm 0cm 0.5cm]{images/Cairns2003.pdf}
\caption[Theoretically predicted radio burst fluxes]{Theoretical flux of the fundamental and harmonic emission bands of a type II radio burst using three wave interaction and stochastic growth theory \citet{cairns2003}.}
\label{fig:cairns_emissivity}
\end{center}
\end{figure}
%Combining elements of the stochastic growth model and the three-wave interaction process allows a calculation of the mount of energy that ends up in the electromagnetic waves, and consequently 
From SGT and the three-wave process, the volume emissivity of the emission \citep{robinson1993a, robinson1998}
\begin{equation}
j_M(r) \approx \frac{\Phi_M}{\Delta\Omega_M}\frac{n_b m_e v_b^3}{3l(r)}\frac{\Delta v_b}{v_b}
\label{eqn:plasma_emiss}
\end{equation}
here the $M$ stands for either fundamental $F$ or harmonic $H$ emission. $\Delta\Omega_M$ is the solid angle over which the  emission is spread, $n_b$ is the electron beam number density, $v_b$ is the beam speed, $l(r)$ is the distance from emission point to observer, $\Delta v_b$ is the width of the beam in velocity space. $\Phi_M$ are known as the conversion efficiencies and are different for fundamental and harmonic emission, see Appendix~\ref{app:emissivity}.

These expressions have been used to simulate radio burst flux resulting in $\sim10^{-17}$\,W\,m$^{-2}$\,Hz$^{-1}$, which have been compared to type II and III radio bursts \citep{schmidt2012, knock2001}. \citet{knock2003} and \citet{cairns2003} used the theory to predict interplanetary type II burst flux as a function of a variety of shock parameters e.g., shocks speed Fig.~\ref{fig:cairns_emissivity}.

The theory outlined in section~\ref{sec:mhd_shocks} -- \ref{sec:three_wave} was employed in a model developed by \citet{schmidt2012b}, completely describing the generation of radio bursts from CME driven shocks. It involves the eruption of a flux rope into a background corona, the driving of a shock as described by the MHD Rankine-Hugoniot relations, the generation of electron beams via SDA, the growth of the bump-on-tail instability, and the generation of electromagnetic emission via the three-wave process and stochastic growth theory. The model finds that that radio emission is generated on the expanding flanks of a CME. The model also shows that when the shock is rippled, there will be spatially intermittent regions of electron beam generation, which could be possible explanation of herringbone emission (Figure~\ref{fig:herbone_model}).
% as shown in Figure~\ref{fig:schmidt_model}
%\begin{figure}[t!]
%\begin{center}
%\includegraphics[scale=0.7]{schmidt_model}
%\caption[Model of radio burst driven by expanding CME flanks]{Complete model of radio burst driven by expanding CME flanks}
%\label{fig:schmidt_model}
%\end{center}
%\end{figure}
\begin{figure}[t!]
\begin{center}
\includegraphics[scale=1.2]{herbone_model}
\caption[Model of radio burst driven by expanding CME flanks]{Model of the radio emission driven by electron beams on the flank of a CME. Each arrow marks the trajectory of the electron beams accelerated at the rippled shock surface. The red and blue lines outline the shock ramp. The green-red colours show radio emission intensity.}
\label{fig:herbone_model}
\end{center}
\end{figure}


\subsection{Frequency Drift of Radio Bursts}\label{sec:freq_drift}

In the previous section it was shown that when plasma emission is excited in the corona, the frequency of emission is close to the plasma oscillation frequency given by equation~\ref{eqn:plasma_frequency}
%As mentioned in section X, the frequency of emission allows a direct conversion from frequency to electron density of the environment in which the emission is generated. Since electron number density in the corona generally varies with height, it is then possible to calculate the height from which this frequency of emission came. To do this, a number of density models of the corona may be used, inclduing the Newkirk, the Saito, or the Baumbach-Allen. These provide a general description of the variation of electron density with height in the corona see Fig.~\ref{fig:density_models.}
As the exciter of the plasma emission moves to greater heights in the corona it will produce plasma emission at continually decreasing frequency due to the dropping density. For example a typical signature of a coronal shock wave in dynamic spectra is two narrow emission lanes (at $f_{plasma}$ and $2f_{plasma}$) drifting toward lower frequency as time passes. A type III radio burst is excited by a much faster source, hence its frequency drift is much faster in dynamic spectra. Generally different types of radio burst have different morphologies when viewed in dynamic spectra owing to their movement (or lack of movement) into regions of different density in the corona. A summary of the burst types is given in Figure~\ref{fig:radiobursts}
\begin{figure}[t!]
\begin{center}
\includegraphics[scale=0.35, trim=2cm 0cm 0cm 2cm]{radiobursts.png}
\caption[Solar radio burst morphologies]{The characteristic shapes of various solar radio bursts seen in dynamic spectra. Type II radio burst are a signature of shock transit from high number density into low number density, as indicated by their drift from high to low frequency. The presence of two distinct bands is indicative of radio emission at the plasma frequency (fundamental) and its second harmonic. These two bands may also be further split themselves, a feature known as band-splitting.}
\label{fig:radiobursts}
\end{center}
\end{figure}

Generally, the frequency drift of the radio burst depends on the velocity of the exciter and the density variation in the corona. It is possible to estimate the speed of the exciter from a set of frequency time values $(f_i, t_i)$. Such as set of values are a direct diagnostic of the number density of the environment of emission versus time e.g., using $f \approx 9000\sqrt{n_e}$, frequency versus time gives number density versus time, $(f_i, t_i) \rightarrow (n_i,t_i)$.
Now, in general the number density of the corona is inversely proportional to height, varying according to some exponential decrease. A very simple model is the hydrostatic case $n(r) = n_0\mathrm{exp}(-r/H)$
\begin{figure}[t!]
%\begin{center}
\begin{minipage}[]{0.5\linewidth}
\centering
\includegraphics[trim= 2cm 0.0cm 0.0cm 2.0cm, scale=0.4]{density_models.pdf}
\end{minipage}
\hspace{0.1cm}
\begin{minipage}[]{0.3\linewidth}
\includegraphics[scale=0.4, trim= 0cm 0.0cm 0.0cm 2.0cm]{1st_harmonic.pdf}
\end{minipage}
\caption[Various models for electron number density in the corona]{(Left)\,Electron number density as a function of height as defined by each of the models listed. The Newkirk, Baumbach Allen, and Saito models each describe an equatorial quiet corona electron density profile. The Saito Eq Hole and Pol Hol describe equatorial and polar hole profiles, respectively. These models are generally used to specify the height of a radio burst in the corona, and the choice of model is often arbitrary. (Right) The first harmonic of the plasma frequency as a function of height defined by the models.}% Added text here to elaborate differences between models
\label{fig:frequency_model}
\end{figure}
where $r$ is distance from sun center, and $H$ is the scale height given by $H = kT/mg$, where $k$ is Boltzman's constant, $T$ is the coronal temperature, $m$ is proton mass and $g$ is gravity. Generally using this $n(r)$, the set of values $(n_i,t_i)$ can be converted to as set of $(r_i,t_i)$
\begin{equation}
(f_i, t_i) \rightarrow (n_i,t_i) \rightarrow n(r) \rightarrow (r_i,t_i).
\end{equation}
Hence, by some appropriate choice of coronal density model $n(r)$, a set of frequency and time values may be converted to a set of height vs time values, from which the velocity may be derived.
%The derived shock velocity firstly assumes purely radial propagation of the shock, which is most likely not the case, and more importantly the method relies heavily on the choice of density model $n(r)$. 
Common practice is to use an $n(r)$ that is derived semi-empirically. For example, there exists a set of models that describe the density fall off with height of the equatorial quiet corona, coronal holes, and active regions \citet{allen1947, newkirk1961, saito1977}, shown in Figure~\ref{fig:frequency_model}. The choice of model affects both the resulting height of the radio emission and the derived speed. A much better diagnostic of radio bursts would be to use actual density measurements of the corona in places of these models, but this is generally not the common practice.

%In Figure 3.5 (left), there is a noticeable difference in both the range of densities and rate of decrease between each of these models. The choice of model to be used in the analysis of type II events is usually determined by the conditions that the shock is assumed to be propagating in. For example there could be propagation in an equatorial or polar coronal hole, through a quiet coronal background, or near an active region or streamer; the electron density is also known to vary with the cycle \citep{lamy2002}. 

%The choice of model affects both the resulting height of the radio emission and the derived speed. Also, even if the shock is known to be propagating in quiet coronal densities there is still a choice between a variety of models, such as the Saito, Newkirk and Baumbach-Allen models, each of which describe quiet equatorial corona. There is a noticeable difference between these models and the choice of one or the other, which is often arbitrary, will effect the resulting shock properties. Ultimately, the analysis of radio bursts are very heavily dependent on the coronal density model that is used, and there is no reason to favor one model over another.












