%!TEX root = ../thesis.tex
%Adding the above line, with the name of your base .tex file (in this case "thesis.tex") will allow you to compile the whole thesis even when working inside one of the chapter tex files


\chapter{Observation and Instrumentation} 
\label{chap:3}

\section{Thompson Scattering Theory}\label{sec:1}

\subsection{Thomson Scattering in the Corona}\label{sec:10}

The first evidence for the existence of the corona was through observations during solar eclipses. The occultation of the solar disk by the moon revealed a visible outer atmosphere structured into streamers and plumes and extending far from the solar surface (Fig.~\ref{fig:eclipse}). This is known as the white-light corona and is due to Thomson scattering of photospheric light by free electrons in the corona.
\begin{figure}[t!]
\begin{center}
\includegraphics[scale=0.45]{images/solar_eclipse}
\caption[White-light corona during an eclipse]{The white-light corona during a solar eclipse. Occultation of the bright solar disk by the moon reveals the faint outer atmosphere of the Sun, known as the corona. It is highly structured, showing features like streamers and plumes. {\it Eclipse photograph courtesy of Miloslav Druckm\"{u}ller \href{http://www.zam.fme.vutbr.cz/~druck/Index.htm}{http://www.zam.fme.vutbr.cz}}}
\end{center}
\label{fig:eclipse}
\end{figure}

The tangential component ($I_T$), radial component ($I_R$), and polarization ($I_P$) of the scattered intensity are given by the expressions
\begin{equation}
I_T=I_0\frac{\pi \sigma_e}{2z^2}[(1-u)C +uD]
\end{equation}

\begin{equation}
I_P=I_0\frac{\pi \sigma_e}{2z^2}\mathrm{sin}^2\chi[(1-u)A +uB]
\end{equation}
with $I_R = I_T-I_P$. $A$, $B$, $C$, and $D$ are the van de Hulst coefficients and are a trigonometric function only of the solid angle subtended by the Sun at the scattering point (see Appendix). $I_0$ is incident intensity, $\sigma_e$ is the electron scattering cross section, $z$ is the distance from scatterer to observer, $u$ is a limb darkening coefficient, and $\chi$ is the angle between a radial vector from sun centre to the scattering electron and a position vector from observer to the electron. 

The total scattered intensity is given by
\begin{equation}
I_{tot} =  2I_T - I_p \sim I_0\frac{\pi \sigma_e}{z^2}\bigg(1 - \frac{\mathrm{sin}^2\chi}{2}\bigg)
\end{equation}


%%%%%% Van de Hulst Coefficients %%%%%%%%%%%%


The van de Hulst coefficients are solutions of a set of integrals to obtain the brightness of each component of the radiation scattered by a single electron in the solar corona. They are a result of scattering theory applied to the case of an electron receiving radiation from the entire solar disk, as opposed to a simpler point source of incident radiation. They are as follows

\begin{subequations}
\begin{align}
\tag{a}
& A = \cos \Omega \sin^2 \Omega \\
\tag{b}
& B = -\frac{1}{8}\bigg[1 - 3\sin^2\Omega -\frac{\cos^2\Omega}{\sin\Omega}(1+3\sin^2\Omega)\textrm{ln}\bigg(\frac{1+sin\Omega}{\cos\Omega}\bigg)\bigg] \\
\tag{c}
& C = \frac{4}{3} - \cos\Omega - \frac{\cos^3\Omega}{3} \\
\tag{d}
& D = \frac{1}{8}\bigg[5 + \sin^2\Omega -\frac{\cos^2\Omega}{\sin\Omega}(5-\sin^2\Omega)\textrm{ln}\bigg(\frac{1+\sin\Omega}{\cos\Omega}\bigg)\bigg] 
\end{align}
\end{subequations}

where $\Omega$ is the angle between the lines QS and QT. Q is the scattering point, S is Sun center, and T is the point where the scattered point vector crosses the Sun at a tangent \citep{howtap2009}.

\begin{figure}[h!]
\begin{center}
\includegraphics[scale=0.3, angle=0]{images/Omega}
\caption[Coronal Thomson scattering geometry]{Geometry of single electron scattering in the solar atmosphere, with angles $\Omega$ and $\chi$.}
\end{center}
\end{figure}


\subsection{White-light observations of CMEs}\label{sec:11}


\section{Coronagraphs}
Before the early 20th century the only way to view the corona was for a short period during a solar eclipse when the moon blocks direct photospheric light. Under normal conditions direct sunlight overwhelms the faint corona. In 1939 the French Astronomer Bernard Lyot developed a telescope, known as a coronagraph, which allowed observation of the corona at any time \citep{lyot1939}. A coronagraph is an optical system that provides an artificial eclipse of direct photospheric light so the much fainter corona can be imaged.
%A coronagraph is an optical system that provides an artificial eclipse so the faint corona may be imaged at any time. The first working coronagraph was invented by the French astronomer Bernard Lyot in 1930.

\subsection{Lyot Coronagraph}\label{sec:22}
The Lyot Coronagraph is the name given to the first optical design of a coronagraph developed by Bernard Lyot. A basic schematic of the instrument is given in Figure~\ref{fig:lyot}. The optical element O1 is a lens that is extremely polished to prevent scattering and reflections of incident light. O1 creates an image of the Sun onto its focal plane at F1 where the occulting disk, D1, reflects away the unwanted solar disk image. F1 then images the objective lens and the occulting disk onto the plane of O2. Lyot's key invention was the Lyot stop and Lyot spot. These are devices onto which light diffracted at the occulting disk is directed and subsequently blocked from being imaged by the final lens O2. O2 then images the faint corona and occulting disk onto the detector plane. 
\begin{figure}[!t]
\begin{center}
\includegraphics[trim=0cm 1.0cm 0cm 0cm, scale=0.8]{images/Lyot_coronagraph}
\caption[The Lyot coronagraph]{A schematic of the basic optical design of the Lyot coronagraph. Lyot's key inventions where the placement of a Lyot stop and Lyot spot at the positions where diffracted light would contaminate the image and obscure the faint corona.}
\label{fig:lyot}
\end{center}
\end{figure}

The Lyot coronagraph is described as internally occulting due to the placement of the occulting disk behind the first objective lens. This is to distinguish it from a externally occulted system in which the disk is placed in front of the objective lens. Modern coronagraphs follow the same basic design of Lyot's but contain extra features such as baffles to stop any scattered light in the telescope. 

\subsection{STEREO COR1 and COR2}\label{sec:22}
The \emph{Solar Terrestrial Relations Observatory} \citep[\emph{STEREO};][]{kai08} Ahead and Behind are two nearly identical spacecraft traveling ahead and behind Earth in its orbit. Each spacecraft is receding from Earth at a rate of $\pm22^{\circ}$ per year, such that they are effectively traveling around the Sun in opposite directions. They carry an identical set of instruments known as the Sun Earth Coronal Connection and Heliospheric Investigation (SECCHI) suite, including in situ detectors and a variety of imagers. On each spacecraft there are two coronagraphs, COR1 and COR2 \citep{how08}. The Ahead COR1 and COR2 combined with Behind COR1 and COR2 offer a stereoscopic view of the corona and any transient event taking place, such as a CME.
\begin{figure}[!t]
\begin{center}
\includegraphics[width=0.8\textwidth]{images/COR1_design}
\caption[The COR1 coronagraph]{A schematic of the basic optical design of the the COR1 coronagraph. There are two such identical instruments, one on the Ahead and one on the Behind spacecraft. It is the same basic design as the Lyot coronagraph with the addition of baffles to prevent scattered light and a polarizer behind the Lyot stop \citep{thomp2008}.}
\label{fig:COR1_design}
\end{center}
\end{figure}
\begin{figure}[!t]
\begin{center}
\includegraphics[width=0.8\textwidth]{images/cor2}
\caption[The COR2 coronagraph]{A schematic of the basic optical design of the COR2 coronagraph. This is an externally occulted coronagraph, meaning it has an extra occultation disk in front of the objective lens. This results in less internally scattered light, but also results in an obscuration of the inner corona. As with COR1, there are two such identical instruments, one on the Ahead and one on the Behind spacecraft \citep{how08}.}
\label{fig:cor2}
\end{center}
\end{figure}


COR1 is an internally occulted Lyot coronagraph, see Figure~\ref{fig:COR1_design}. It images the inner corona with a field of view from $1.4 - 4.5\,R_{\odot}$ in a waveband 22.5\,nm wide centered on the H$\alpha$ line at 656\,nm. It has an internal polarizer that takes three images at $0^{\circ}$, $120^{\circ}$, and $240^{\circ}$, so that polarized or total brightness images of the inner corona may be produced. It nominally produces $1024\times1024$ pixel images with platescale of 3.75 arcsec per pixel \citep{thomp2008}. A typical observing sequence will give an image cadence of 10 minutes.

COR2 is an externally occulted Lyot coronagraph. Externally occulted coronagraphs have an extra occulting disk in front of the objective lens, see Figure~\ref{fig:cor2}. This is to prevent direct sunlight scattering off of the objective lens, making internally scattered light less of a problem for this type of coronagraph. A downside to this design is that the external occulter does not allow the inner corona to be imaged, hence such coronagraphs are usually used to observe the extended corona to larger heights. COR2 observes the corona in a field of view from $2.5 - 15\,R_{\odot}$ and in a wavelength range of $650 - 750$\,nm. It nominally produces $2048\times2048$ images, with 14.7 arsec per pixel. Like COR1 it has an internal polarizer producing three linearly polarized images per observing sequence (30 minutes).

These white light imagers of the corona allow for a stereoscopic view of CMEs in a total field of view covering $1.4 - 15\,R_{\odot}$. The two viewpoint capabilities of these telescopes offer a more accurate observational estimation of both CME kinematics and CME mass, resulting in a better understanding of CME dynamics.

\subsection{SOHO LASCO}\label{sec:23}

The Large Angle Spectroscopic Coronagraph (LASCO) on board the Solar and Heliospheric Observatory (SOHO) comprises three different coronagraphs, C1, C2, and C3, that were built to observe the corona over heliocentric distance range of 1.1--30\,$R_{\odot}$ \citep{bru95}. 

C1 is a mirror version of the internally occulted Lyot coronagraph that images the corona over a distance range of $1.1- 3\,R_{\odot}$. A tunable Fabrey-Perot interferometer is placed after the Lyot stop to allow narrow passband images of the solar corona at the lines of Fe\rmnum{14}, Ca\rmnum{15}, Na\rmnum{1}, Fe\rmnum{10}, and H$\alpha$,
C1 operates a $1024\times1024$ CCD with a pixel size of $5.6"$, giving a spatial resolution of $\sim11"$. The Rayleigh diffraction limited resolution of the telescope if $3.3"$ at $530.3$\,nm. Unfortunately C1 failed in 1997 and is no longer in operation. 
\begin{figure}[!t]
\begin{center}
\includegraphics[scale=0.4]{c2_concept.png}
\caption[The LASCO C2 coronagraph optics]{Conceptual optical layout of the LASCO C2 corongraph. The top ray-tracing diagram shows the image formation, while the bottom diagram demonstrates the stray light supression and occultation \citep{bru95}.}
\label{fig:c2_concept}
\end{center}
\end{figure}

C2 is an externally occulted, broadband, lens coronagraph that images the corona from $1.5-6\,R_{\odot}$\footnote{The levels of stray-light just beyond the occulter mean that in practice the C2 inner field of view limit is $2.2\,R_{\odot}$}. The conceptual diagram for its optical layout is shown in Figure~\ref{fig:c2_concept}. The front aperture A0 is on the left of the diagram and includes the external occulter D1, consisting of three consecutive circular disks on a common spindle, these disks completely shadow the entrance aperture A1 from direct sunlight. The disks also successively intercept diffracted sunlight from the one before to minimize the total diffracted light from the solar disk falling on the object lens O1. 
\begin{figure}[!t]
\begin{center}
\includegraphics[scale=0.5, angle=90]{c2_detail.png}
\caption[The LASCO C2 coronagraph detail]{A detailed layout of the C2 coronagraph optics and mechanical construct \citep{bru95}.}
\label{fig:c2_detail}
\end{center}
\end{figure}
The disks have fine threads on their edges that have been diamond machined, and have a cone opening angle that is slightly bigger than the solar disk at L1 ($32.31^"$). This ensures a very high level of light rejection at the external occulter. The objective lens O1 images the corona at the field stop, which sets the outer limit to the field of view of the telescope ($5\,R_{\odot}$ for C2). O1 also images D1 onto the D2 internal occulter, which also serves to intercept any residual stray light diffracted at D1. The lens O2 collimates the coronal image and also images the entrance aperture A1 onto the Lyot stop A3. Finally the lens O3 images the corona onto the focal plane F. The combination of the internal and external occulters, the aperture stops, and a heat rejection mirror at the entrance aperture (Figure~\ref{fig:c2_detail}), mean that the the coronagraph has stray light levels that are an order of magnitude improvement over previous space based coronagraphs e.g., stray light levels are typically an order of magnitude or more lower than the coronal brightness, allowing the detection of features in the range of $2\times10^{-7} - 15\times0^{-10}\,\mathrm{B}_{\odot}$ where $\mathrm{B}_{\odot}$ is mean solar brightness (MSB) unit \citep{bru95}.

The light focused onto the focal plane by the O3 lens first pases through colour filter and polarizer wheel, which selects the band pass and takes three polarization images at $0^{\circ}$, $-60^{\circ}$ and $+60^{\circ}$ such that total or polarized brightness images of the corona may be taken. Mounted at the focal plane is a front-side illuminated $1024\times1024$ CCD, with $21\mu$ square pixels (equivalent angular size if 11.4"),  operating in a nominal spectral range of $500-700$\,nm, at a temperature of $-80^{\circ}$\,C. CCD readout takes approximately 22 seconds, after which the image may be compressed by a number of algorithms before being sent to the ground station. Among the observing nominal modes are routine flat-field exposures and dark current exposures to be used in calibration. At current operations the C2 coronagraph offers a total brightness image of the corona every 15 minutes.  


Finally, C3 is an externally occulted broadband lens coronagraph that images the corona from $3.7-30\,R_{\odot}$, with a pixel size of $56"$ and brightness range of $3\times10^{-9} - 1\times10^{-11}$. It has the same optical design as C2 (Figure~\ref{fig:c2_concept}).

\section{Radio Spectrometers and Radioheliographs}\label{sec:3}



\subsection{RSTO Callisto}\label{sec:30}

\subsection{STEREO WAVES}\label{sec:31}

Below this frequencies of $\sim$10\,MHz the Earth's ionosphere prevents the propagation of radio waves, hence radio emission below this frequency cannot be observed from ground-based observatories. To overcome this problem, there are a number of space based instruments that observe frequencies in the deca and kilometric wavelength ranges; the STEREO A and B spacecraft each carry such an instrument, known as STEREO WAVES or S/WAVES \citep{bougeret2008}. This instrument directly inherits it's design from the Ulysees \citep{stone1992}, Wind \citep{bougeret1995}, and Cassini \citep{gurnett2004} spacecraft.

The antenna system on each STEREO spacecraft consists of three mutually orthogonal 6-meter Beryllium-Copper monopole elements \citep{bale2008}, Figure~\ref{fig:swaves_antennas}. The elements diameter at the base is 1\,inch and tapers to 0.6\,inches at the tip. The three monopoles are each connected to a low noise and high impedance preamplifiers. The three preamplifiers each feed a number of receiving systems including the Fixed Frequency Receiver (FFR), the Time Domain Analyzer (TDS), and two frequency domain analyzers known as the Low Frequency Receiver (LFR) and High Frequency Receiver (HFR); only the LFR and HFR are described here.
\begin{figure}[!t]
\begin{center}
\includegraphics[scale=0.5]{Stereo-antennas.png}
\caption[The SWAVES antennas]{Configuration of the STEREO/WAVES antennas as seen from the Sun-Earth line, facing the Earth. Each monopole is 6\,m long and feeds a number of receiving instruments.}
\label{fig:swaves_antennas}
\end{center}
\end{figure}

The LFR is s direct conversion receiver that performs dynamic spectral analysis in the 2.5--160\,kHz band. The signal fed from the antennas and pre-amp is passed through a wavelet-like transform to perform digital spectral analysis in three 2-octave bands which each have 16 logarithmically spaced frequency channels (resulting in 48 log spaced channels in the 2.5--160\,kHz range). Receiver automatic gain control (AGC) ensures a high dynamic range of 120\,dB. Various combinations of the three antenna elements can be made to produce pseudo-dipole or monopole configurations. The signal is then 12-bit digitized and sent to a Digital Processing Unit (DPU)

The HFR is a dual sweeping receiver operating in the frequency range 125 kHz--16.025 MHz. It uses a super-hetereodyne technique to down-convert the signal frequencies to odd multiples of 25\,kHz that specifically avoid lines of noise that are generated by the spacecraft power supply, which may produce harmonic multiples of 50\,kHz. Due to the down-conversion process the HFR frqeuency range is covered in multiple steps of 50\,kHz increments, thus the highest spectral resolution is 50\,kHz. It has a dynamic range of 80\,dB. Like the LFR, it can receive various combinations of the three antenna elements in dipole or monopole modes. The signal is digitized by the same DPU as the LFR.

The time domain resolution for both the LFR and HFR is determined by the DPU. Nominal time resolution of the dynamic spectra produced by both receivers is 1\,minute.

\subsection{Nan\c{c}ay Decametric Array}\label{sec:32}

The Nan\c{c}ay Decametric Array consists of 72 ($6\,\mathrm{east} \times12\,\mathrm{west}$)conical antennas consisting of a left-handed helically wound component and a right handed helical component -- this makes the array contain 144 antennas in total, comprising 8000\,m$^2$ effective aperture at 30\,MHz \citep{lecacheux2000}. Each helix antenna is made of eight copper-steel wires wound on the surface of a cone and connected to the output coaxial cable by diode switches; only six wires are used at a time to form the antenna; the other two, diametrically opposite, are left disconnected. By changing the connections through the diode switches, the antenna can be electrically rotated around the cone axis, corresponding to a phase change of the antenna by steps of 45$^{\circ}$. The antennas are broadband (20--120\,MHz), low gain (low directivity) with a half power beam width of 90$^{\circ}$ centered on the cone axis \citep{boischot1980}. The entire array is steerable by phase delays withtin the 90$^{\circ}$  beam width of the individual antennas resulting in a possible tracking time of $\pm$4\,hr around the meridian, within a declination range of $-20^{\circ}$ to $+50^{\circ}$ .

The backend of the array consists of three possible receivers. The decametric activity consists of a wide-band swept-frequency analyzer which operates 400 channels between 20-90\,MHz. Left and right hand circular polarization are alternately sampled every 0.5 seconds from the left and right hand helical feeds of the antennas. The nominal operations for solar radio burst monitoring use this swept frequency receiver but there are other more sophisticated receivers available such as spectro-polarimeter with 1\,ms time sampling over providing 1024 channels and with a 60\,dB dynamic range. However, it has only a 12.5\,MHz instantaneous band, which is not so useful for solar radio burst monitoring.

\subsection{Nancay Radioheliograph}\label{sec:33}

Radio images of the solar corona are important to probe both quiet activity and also flaring an eruptive activity. As described in Section 2.X, flaring and eruptive activity may often result in high speed electrons which can excite plasma emission and gyrosynchrotron emission. The Nan\c{c}ay Radioheliograph is a solar-dedicated radio interferometer located at Nan\c{c}ay, central France ($47^{\circ}$N~$2^{\circ}$E), that observed such activity at ten frequencies between 150 and 450\,MHz \citep{kerdraon1997}.
\begin{figure}[t!]
\begin{center}
\includegraphics[scale=0.55, angle=90]{nrh_layout}
\caption[The Nancay Radioheliograph layout]{The Nancay Radioheliograph layout}
\end{center}
\label{fig:eclipse}
\end{figure}

The array antennas are arranged in a perpendicular \textquoteleft T' shape. The east west array consisting 19 antennas providing baselines in the range of 50\,m to 3200\,m. Four these antennas have parabolic collectors with four orthogonal thick dipole feeds at the focus providing two orthogonal polarizations in the 150-450\,MHz band. The remaining 15 have no collectors (no dishes) and consist only of thick dipole antennas providing linear polarization only. The north-south array consists of 24 five meter dishes with wide band feeds, covering baselines between 54 to 1248\,m. The antenna front end electronics include low noise high dynamic range (45\,dB) pre-amplifiers, band filters for frequency switching and a local oscillator which mixes the signal 113\,MHz before it is sent to the receiver \citep{avignon1989}. At the receiver the signal is further mixed down to 10.7\,MHz and fed through a bandpass filter of 700\,kHz width (final bandwidth of each observed frequency), digitized and sent to the correlator.

%Mark II: monofrequency east-west (169 MHz)
%Mark III: monofrequency east-west and north-south (169 MHz), receiver has 45 dB dynamic range (Bonmartin 1983)
%Mark IV: multifrequency (5) on north-south (150-450 MHz), monofrequency on the east-west (164 MHz) (Avignon 1989)
% Multifrequency (5) on north-south and east-west baslines (150-450 MHz) (1993 AdSpR)
%Present: multifreqquency (10) on north-south and east-west baselines (150-450 MHz), digital correlator added (Kerdraon 1997)

The original array consisted of only the east-west baselines, with the north-south being added later in the early 1980s \citep{bonmartin1983}, hence the two separate arrays operate off different correlators. The visibility outputs of each correlator are digitized by sampling every 5\,ms, resulting in 4 images every 5\,ms (Stokes I and V for the east-west and north south arrays), with a down sampling of by integrating at least 4 successive images in order to ease storage loads \citep{avignon1989}. These images are only 1D, offering projected intensity profiles along two axes. Originally, full 2D maps were created by Earth rotation synthesis using the standard 1D observations over one entire day \citep{nrh1993}. Such limitations were mainly due to the use of an analog correlator. However, a digital correlator installed in 1997 now provides fast 2D images using the most westerly 17 antennas of the east-west baseline \citep{kerdraon1997}, resulting in a spatial resolution that is 4 times lower than the 1D east-west images. Systematic daily observations of 2D images are usually performed at between 0.1 and 1 image per second.

Instrument phase and gain calibration is performed by observing Cygnus A. Phase and gain accuracy for these measurements are $5^{\circ}$ and 5\%, respectively, with only 1 calibration per week necessary because of good system stability \citep{avignon1989}

\begin{tabular}{ |l|l| }
  \hline
  \multicolumn{2}{|c|}{Nancay Radioheliograph properties} \\
  \hline
  Time resolution & 5\,ms (integrated to 0.1 -- 1\,s for 2D images) \\
  Spatial Resolution & 0.3 -- 6\,arcmin, depending on frequency and direction \\
  Observing frequencies & 10 frequencies between 100 -- 450\,MHz \\
  Bandwidth & 700\,kHz \\
  Polarization & Stokes I and V \\
  Observing time & 7.5\,hr centered around 12 UT \\
  \hline
\end{tabular}
%The Mark IV Nancay Radioheliograph Solar Physics 120 p193) 
%Correlation is performed independently for the EW and NS baselines





\section{EUV imaging}\label{sec:4}

\subsection{SDO AIA}\label{sec:40}



