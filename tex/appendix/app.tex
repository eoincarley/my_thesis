%!TEX root = ../thesis.tex
%Adding the above line, with the name of your base .tex file (in this case "thesis.tex") will allow you to compile the whole thesis even when working inside one of the chapter tex files

\chapter{A Nice Appendix}
\label{app:1}

If we assume the herringbones are due to the shock drift acceleration process then the velocity upon a reflection from the shock is
\begin{equation}
v_{r,||} = 2v_{shock}\mathrm{sec}\,\theta_{Bn} - v_{i,||}
\end{equation}
where $v_{r,||}$ is the reflected parallel velocity of the particle, $v_{i,||}$ is the incident parallel velocity of the paticle, $v_{shock}$ is the shock velocity, and $\theta$ is the angle between upstream $B-$field and shock normal $n$. Taking the shock speed to be the speed of the 150\,MHz radio source, $550\times10^3$\,m\,s$^{-1}$, and $v_{i,||}$ to be the thermal speed of an electron 
\begin{equation} 
v_{thermal} = \sqrt{ \frac{3k_bT}{m_e} }
\end{equation}
At $1\times10^{6}$\,K, $v_{thermal} = v_{i,||} = 6.7\times10^6$\,m\,s$^{-1}$. Now, the herringbone electron speed 0.15\,c, this is the reflected speed $v_{r,||}$ in equation A1. Rearranging A1 we get
\begin{equation}
\theta_{Bn} = \mathrm{sec}^{-1}\bigg( \frac{1}{2}\frac{v_{r,||} +  v_{i,||} }{v_{shock}}\bigg)
\end{equation}
Substituting the above values we get $\theta_{Bn}=88^{\circ}$. Independent verification of a quasi-perpendicular shock orientation!


